
{\selectlanguage{polish}
\begin{abstract}

Tematem rozprawy doktorskiej są teoretyczne badania wpływu oddziaływań wielociałowych na własności statyczne i dynamiczne  zerowych modów Majorany, zrealizowanych w układach bezspinowych fermionów opisanych zmodyfikowanymi modelami Kitaeva.

Jednym z najważniejszych wyników rozprawy jest algorytm umożliwiający identyfikację zerowych modów Majorany dla dowolnego hamiltonianu, również z oddziaływaniami wielociałowymi.
Algorytm  przetestowano, w pierwszej kolejności porównując otrzymane wyniki  z wynikami dostępnymi w literaturze.
Następnie sprawdzono w jaki sposób oddziaływania wielociałowe oraz ich zasięg wpływają na czasy życia zerowych modów Majorany oraz na ich strukturę przestrzenną.
Zwiększanie oddziaływań wielociałowych prowadzi do zmniejszenia stabilności zerowych modów Majorany.
Co więcej, wraz ze wzrostem zasięgu oddziaływań, rośnie destruktywna rola tych oddziaływań na czas życia modów Majorany.

Przedstawiono również implementację konstrukcji nowej bramki fazowej dla qubitu bazującego na zerowych modach Majorany.
W odróżnieniu od standardowej koncepcji wspomnianej bramki, która bazuje na fazie dynamicznej, zaprezentowana bazuje na fazie geometrycznej.
Działanie bramki polega na podwójnym wyplataniu przekrywających się zerowych modów Majorany.
Pokazano, że faza tej bramki zależy od wszystkich parametrów hamiltonianu, a w szczególności od tematu pracy --- oddziaływań wielociałowych.


Poza przedstawieniem wyników i ich analizą, rozprawa doktorska posiada rozbudowany wstęp teoretyczny, związany ze wszystkimi omawianymi zagadnieniami,
a wszelkie przydatne wyprowadzenia można znaleźć w załączniku do rozprawy.


\end{abstract} 

\ornament


\noindent\small
\textbf{Słowa kluczowe} \quad zerowe mody Majorany \quad oddziaływania wielociałowe \quad model Kitaeva \quad wyplatanie kwantowe \quad \\topologiczny komputer kwantowy
}


\newpage



{\selectlanguage{english}
\begin{abstract}
This dissertation concerns the theoretical studies of the influence of many-body interaction on static and dynamic properties of Majorana zero modes, realized in fermionic spinless systems, which can be described by the Kitaev model.

The algorithm for Majorana zero modes identification has been presented in the thesis.
The algorithm works for any Hamiltonian, also when many-body interactions are present in the system.
First, the algorithm has been tested by comparing the results to the literature.
In the next step, the influence of strength and range of the interaction on Majorana zero modes lifetimes and their spatial structures has been investigated.
Increasing many-body interaction strength leads to decreasing Majorana zero modes stability.
Moreover, if the range of the interaction increases, the destructive role of the interaction on Majorana zero modes lifetimes also increases.

The implementation of the new phase-gate for qubit based on Majorana zero modes was also presented.
Unlike the standard phase gate implementation, which is based on the dynamic phase, the presented gate depends on the geometric phase.
The protocol of the gate consists of the double braiding of two non-overlapping Majorana zero modes.
It has been shown that the phase of this gate does depend on all Hamiltonian parameters, especially on the thesis major --- many-body interactions.

In addition to presenting results and their analysis, the dissertation contains expanded theoretical introduction, related to all discussed problems.
All useful derivations and proofs can be found in the Appendix attached to the thesis.
\end{abstract}

\ornament


\noindent\small
\textbf{Key words} \quad Majorana zero modes \quad many-body interactions \quad Kitaev model \quad quantum braiding \quad\\ topological quantum computer
}

\vfil\vfil






\cleardoublepage