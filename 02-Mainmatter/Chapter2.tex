
\chapter{Zerowe Mody Majorany} \label{chap:majorana}

\section*{Opis rozdziału}

W tym rozdziale czytelnik zostanie wprowadzony w definicję \MZM, zostaną wskazane różnice pomiędzy operatorami Majorany $\gammai$ na węźle $i$, a \MZM\ $\Gammai$.
Przedstawiony tutaj zostanie również najprostszy model realizujący \MZM --- model Kitaeva.
Przedstawiona analiza w tym rozdziale została opracowana na podstawie prac \cite{kitaev.2001,stanescu.2016,wieckowski.maska.2018,sarma.freedman.2015}.



\section{Operatory Majorany}

Operatorami Majorany nazywamy operatory hermitowskie $\gammai$, które spełniają następujące relacje antykomutacji
\begin{equation}
    \{ \gammai, \gammaj\} = 2\deltaij .\label{eq:majoranaCom0}
\end{equation}
Najważniejsze własności tych unitarnych operatorów są następujące:
\begin{align}
\gammaii^\dagger &= \gammai\label{eq:majoranaOperatorHermitian},\\    
\gammai^2 &= \bbone\label{eq:majoranaOperatorOne}.
\end{align}
Wspomniane operatory o takich własnościach, można skonstruować za pomocą $\sites$ par fermionowych operatorów kreacji $\aid$ i anihilacji $\ai$:\footnote{Wprowadzono tutaj indeksy $\pm$, ponieważ z każdym węzłem $i$ możemy zdefiniować dwa różne operatory majorany $\gammai^+$, $\gammai^-$.}
\begin{align}
    \gammai^+ &= \ai+\aid,\label{eq:operatorMajoranaPlus}\\
    \gammai^- &= \iu(\ai-\aid)\label{eq:operatorMajoranaMinus},
\end{align}
gdzie $L$ to liczba węzłów, a operatory $\ai,\aid$ spełniają relacje antykomutacji dla fermionów:
\begin{equation}
\{ \ai,\ajd \} = \deltaij, \,\,\,\{\aid,\ajd\}=0.\label{eq:aidtCommStandard} 
\end{equation}
Tak zdefiniowane operatory są hermitowskie $(\gammai^\alpha)^\dagger=\gammai^\alpha$, a ich kwadrat jest równy operatowi jednostkowemu $(\gammai^\alpha)^2=\bbone$, (gdzie $\alpha=\pm$). Z uwagi na wymienione własności, dla operatorów Majorany $\gammai$  nie istnieje analogiczny operator liczby cząstek $\ni=\aid\ai$  jak w przypadku operatorów kreacji $\aid$ i anihilacji $\ai$.
Można pokazać, że takie operatory spełniają następującą relację antykomutacji
\begin{equation}
    \{ \gammai^\alpha, \gammaj^\beta\} = 2\deltaij \deltaab.\label{eq:majoranaCom}
\end{equation}
Transformacja odwrotna jest następująca:
\begin{align}
    \aid &= \tfrac12 \left(\gammai^++\iu \gammai^-\right),\\
    \ai &= \tfrac12 \left(\gammai^+-\iu \gammai^-\right).
\end{align}
Odwrotna transformacja ma interesującą interpretację. 
Każdy stan fermionowy, $|0\rangle$ pusty albo $|1\rangle$ zapełniony, konstruujemy z dwóch operatorów Majorany (patrz rysunek~\ref{fig:fermionConstruction}).
Zbiór operatorów Majorany stanowi bardzo wygodną bazę operatorów $\majoranaBase=\{\gammai^\alpha\}$, która może zostać wykorzystana do opisu \MZM. 


Ważnym operatorem jest operator parzystości $\parity_i$, który można w bazie operatorów Majorany przedstawić w następującej postaci
\mymarginpar{\footnotesize
\tikz{
\draw[ultra thick,gray](0,0)circle(0.7cm);
\draw[very thick,blue] (-.3,0)--(.3,0);
\draw[very thick,blue,fill=blue](-.3,0)circle(0.15cm);
\draw[very thick,blue,fill=white](.3,0)circle(0.15cm);
\draw[-latex,very thick] (-1,-.81)node[below]{\normalsize$\gammai^+$}--(-.4,-.2);
\draw[-latex,very thick]
(1,-.81)node[below]{\normalsize$\gammai^-$}--(.4,-.2);
\draw[-latex,very thick]
(1,1)node[above]{\normalsize fermion}--(.0,.3);
}
\captionof{figure}{Konstrukcja stanu fermionowego z operatorów Majorany.}
\label{fig:fermionConstruction}
}
\begin{equation}
    \parity_i = 1-2\ni = \iu \gammai^+\gammai^-.\label{eq:parity}
\end{equation}
Operator ma dwie wartości własne $\pm 1$: $+1$ dla stanu nieobsadzonego oraz $-1$ dla stanu obsadzonego.
W późniejszych rozdziałach operator parzystości zostanie wykorzystany do opisu działania qubitu na bazie \MZM.
Wymnażając wszystkie operatory parzystości $\parity_i$ --- takie operatory komutują ze sobą $[\parity_i,\parity_j]=0$ --- można skonstruować całkowity operator parzystości układu
\begin{equation}
    \parity = \prod_i \parity_i.\label{eq:totalParity}
\end{equation}
Taki operator ma dwie wartości własne $\pm1$. Wartości własne numerują stany z parzystą i~nieparzystą całkowitą liczbą cząstek w układzie.\\

Operatory Majorany zdefiniowane w równaniach~\labelcref{eq:operatorMajoranaPlus,eq:operatorMajoranaMinus} stanowią ortonormalny zbiór operatorów  $\majoranaBase$ z iloczynem skalarnym\footnote{Dowód można znaleźć w~dodatku~\ref{chap:derivations}.}
\begin{equation}
    (\gammai^\alpha|\gammaj^\beta) = \deltaij \deltaab,\label{eq:majoranaOrthogonality}
\end{equation}
gdzie $(A|B) = \Tr(AB)/\Tr(\bbone)$ to iloczyn Hilberta--Schmidta operatorów.
Ortogonalność tych operatorów została wykorzystana w późniejszych rozdziałach do zaprojektowania algorytmu do znajdowania \MZM\ w danych układach.
Bazę operatorów Majorany $\majoranaBase$ można rozszerzyć o kombinację operatorów Majorany wyższych rzędów
\begin{equation}
    (\Upsilon^1_{\mathfrak{im}}=\gammaii_{i}^{\,\,m}),\,\,
    \Upsilon^3_{\mathfrak{im}}=\iu\gammaii_{i_1}^{\,\,m_1}
    \gammaii_{i_2}^{\,\,m_2}
    \gammaii_{i_3}^{\,\,m_3}, \,
    \dots, \Upsilon^{2M+1}_{\mathfrak{im}}=\iu^M\gammaii_{i_1}^{\,\,m_1}\cdots\gammaii_{i_{2M+1}}^{\,\,m_{2M+1}}.
\end{equation}
Nieparzysta liczba operatorów w konstrukcji $\Upsilon^{2M+1}$ jest kluczowa.
Jeśli rozszerzalibyśmy bazę $\majoranaBase$ o parzystą kombinację operatorów $\gammai^\alpha$, stracilibyśmy warunek dotyczący ortogonalnych operatorów bazy \eqref{eq:majoranaOrthogonality}.
Natomiast, po rozszerzeniu bazy $\majoranaBase$ o dowolne  kombinacje zawierające nieparzystą liczbę operatorów $\gammai^\alpha$, t.j. $\Upsilon^3_{\mathfrak{im}},\Upsilon^5_{\mathfrak{im}},\dots, \Upsilon^{2M+1}_{\mathfrak{im}}$,  wszystkie operatory $\gammai\in\majoranaBase$ dalej spełniają równania~\labelcref{eq:majoranaCom0,eq:majoranaOrthogonality,eq:majoranaOperatorHermitian,eq:majoranaOperatorOne}, a zatem stanowią ortogonalny zbiór operatorów Majorany.
W rozprawie doktorskiej skupimy się na konstrukcji \MZM\ wyłącznie bazując na bazie skonstruowanej z pojedynczych operatorów $\gammai^m$ ($\Upsilon^1_{\mathfrak{im}}$).
W pracy~\cite{wieckowski.maska.2018} sprawdzaliśmy wyrazy $\Upsilon^3_{\mathfrak{im}}$, ale ich wpływ był znikomy podczas konstrukcji \MZM\ w badanym zagadnieniu.
Wyższe wyrazy $\Upsilon^{2M+1}$, $M>0$, nie są takie istotne do rozwiązywania problemów rozważanych w rozprawie doktorskiej.
Wszelkie poszukiwania \MZM\ w układach przedstawionych w tej rozprawie doktorskiej ograniczone są do poszukiwania kombinacji lokalnych operatorów Majorany~\cite{wieckowski.maska.2018}.

\ornament

\section{Definicja zerowych modów Majorany}\label{sec:MZMdef}
\glslink{MZM}{Zerowe mody Majorany (ang. Majorana zero modes, MZM)} $\Gammai$ to są operatory Majorany $\gammai$, które dodatkowo są całkami ruchu.
Matematyczna definicja  \MZM\ jest następująca~\cite{sarma.freedman.2015,wieckowski.maska.2018,wieckowski.ptok.2019}:
\begin{align}
\Gammai^\dagger &= \Gammai,\label{eq:MZMherm}\\
[\hatH, \Gammai] &= 0,\label{eq:MZMcom}\\
\{\Gammai,\Gammaj\} &= 2\deltaij.\label{eq:MZManticom}
\end{align}
\MZM\ są to operatory hermitowskie \eqref{eq:MZMherm}, komutują z hamiltonianem \eqref{eq:MZMcom}
oraz antykomutują ze sobą~\eqref{eq:MZManticom} --- spełniają identyczne relacje antykomutacji jak operatory Majorany \eqref{eq:majoranaCom0}.
Naturalną bazą dla takich operatorów są operatory Majorany $\gammai$. 
\MZM\ w bazie operatorów Majorany $\majoranaBase$ ma następującą postać
\begin{equation}
    \Gammaii_n = \sum_i \alphai^{n}\, \gammai,
\end{equation}
gdzie $\alphai^n\in\realNumbers$.
W bazie $\majoranaBase$ skonstruowanej z $\sites$ par lokalnych operatorów, \labelcref{eq:operatorMajoranaPlus,eq:operatorMajoranaMinus}, $\Gammaii_n$ ma następującą postać
\begin{equation}
    \Gammaii_n = \sum_{m=\pm} \sum_{i=1}^{\sites} \alphai^{m,n}\, \gammai^m,\label{eq:mzmBase}
\end{equation}
gdzie na współczynniki $\alphai^m$ nakłada się warunki normalizacyjne
\begin{equation}
    \sum_m\sum_i (\alphai^{m,n})^2=1.
\end{equation}
Warunki normalizacyjne są kluczowe --- bez nich nie było by spełnione równanie
\eqref{eq:MZManticom}.

\ornament

\section{Model Kitaeva}\label{sec:kitaev}

Po dość abstrakcyjnym wstępie matematycznym, pora na przedstawienie prostego modelu, w którym można zrealizować \MZM.
Przełomowa praca \textit{Unpaired Majorana fermions in quantum wires}~\cite{kitaev.2001} autorstwa Alexeia Kitaeva (cytowania: 3148, stan na 5 lutego 2020 roku), rozpoczęła okres burzliwego zainteresowania fizyków z całego świata fizyką związaną z tematyką nadprzewodnictwa oraz topologicznych komputerów kwantowych.
Zaprezentowana w~tej sekcji analiza modelu Kitaeva bazuje na oryginalnej pracy Kitaeva~\cite{kitaev.2001}.

Model Kitaeva \cite{kitaev.2001}, to najprostszy model mikroskopowy, który w pewnym zakresie parametrów, może realizować \MZM.
Model ten można opisać za pomocą następującego hamiltonianu
\begin{equation}
    \hatH_{\text{Kitaev}} = 
    \underbrace{\sum_{\langle i,j\rangle}\left(\t0 \, \aid \aj + \hc\right)}_{\hatH_{{\text{kin}}}}+
    \underbrace{\sum_{\langle i,j\rangle}\left(\DeltaSC \aid \ajd+\hc\right)}_{\hatH_{\text{prox}}}
     + \underbrace{\sum_i \mui \ni}_{\hatH_{\muuniform}},\label{eq:kitaevHamiltonian}
\end{equation}
gdzie $\aid/\ai$ to odpowiednio operator kreacji/anihilacji bezspinowego fermionu w węźle $i$, $\t0$~to~całka przeskoku, $\DeltaSC$ to przerwa nadprzewodząca, $\mui$ potencjał chemiczny, $\ni=\aid\ai$ operator liczby cząstek.

W celu pokazania, że model rzeczywiście może zawierać \MZM, rozważymy pewne specjalne przypadki:\footnote{Wyprowadzenie można znaleźć w dodatku~\ref{chap:derivations}.}
\begin{enumerate}
    \item $\t0=\DeltaSC=0,\mui>0$,
Hamiltonian \eqref{eq:kitaevHamiltonian} przyjmie wtedy postać
\begin{equation}
    \hatH_{\text{triv}} = \sum_i \mui \ni = \tfrac12\sum_i \mui\left(1-\iu\gammai^+\gammai^-\right).\label{eq:kitaevHamiltonianTriv}
\end{equation}
Układ jest w \textit{fazie trywialnej} --- brak \MZM\ w układzie --- nie istnieją takie kombinacje~\eqref{eq:mzmBase}, które komutują z hamiltonianem.
Stan podstawowy takiego hamiltonianu będzie jednym ze stanów bazowych przestrzeni Hilberta $\hilbertSpace$ zgodnie z rozkładem $\mui$.\footnote{Dla $\mui>0$ stanem podstawowym będzie stan $|0\cdots0\rangle$.}
    
    \item \label{enum:topo} $\t0=\DeltaSC>0,\mui=0$,
    w tym przypadku hamiltonian \eqref{eq:kitaevHamiltonian} przyjmie postać
\begin{equation}
    \hatH_{\text{topo}} = \iu \sum_{\langle i,j\rangle}\t0\,\gammai^-\gammaj^+.\label{eq:kitaevHamiltonianTopo}
\end{equation}
Załóżmy, że $\sites$ węzłów formuje łańcuch $\langle i,j\rangle \to \langle i,i+i\rangle$ --- jednowymiarowy układ z otwartymi warunkami brzegowymi oraz $\t0 = t^0$ jest jednorodne w układzie.
Taki układ jest w \textit{fazie topologicznej} --- zawiera dwie, zlokalizowane \MZM\ znajdujące się na brzegach układu.
Hamiltonian takiego układu jest następujący 
\begin{equation}
\hatH_{\text{topo}}^{\text{chain}} = \iu\, t^0\sum_{i=1}^{\sites-1}\gammai^-\gammaii_{i+1}^+.\label{eq:kitaevHamiltonianTopoChain}
\end{equation}
W hamiltonianie brakuje operatorów $\gammaii_1^+$ oraz $\gammaii_{\sites}^-$.
Takie operatory oczywiście komutują z hamiltonianem,
są całkami ruchu 
\begin{equation}
[\hatH_{\text{topo}}^{\text{chain}},\gammaii_1^+]=
[\hatH_{\text{topo}}^{\text{chain}},\gammaii_{\sites}^-]=
0,
\end{equation}
a zatem to są \MZM\ w takim układzie: $\Gammaii_1=\gammaii_1^+$, $\Gammaii_2 = \gammaii_{\sites}^-$.
\end{enumerate}
Te dwa przypadki ilustrują dwa sposoby parowania operatorów Majorany, patrz rysunek~\ref{fig:mzmPhases}. 
W pierwszym przypadku operatory Majorany parowane są na tym samym węźle --- mody fermionowe budowane są w obrębie danego węzła.
W drugim przypadku operatory Majorany parowane są na sąsiednich węzłach --- dwa operatory, na brzegach układu są niesparowane, czyli są \MZM. 

\begin{figure}
    \centering
    \begin{tikzpicture}
%lines
\draw (-0.5,1.2)node[right]{(a) faza trywialna};
\draw[ultra thick,gray](0,.1)--(1.5,.1);
\draw[ultra thick,gray,xshift=1.5cm](0,.1)--(1.5,.1);
\draw[ultra thick,gray,xshift=2*1.5cm](0,.1)--(1.5,.1);

\begin{scope}[yshift=-.2cm]
\draw[ultra thick,gray](0,.1)--(1.5,.1);
\draw[ultra thick,gray,xshift=1.5cm](0,.1)--(1.5,.1);
\draw[ultra thick,gray,xshift=2*1.5cm](0,.1)--(1.5,.1);
\end{scope}

%balls
\draw[ultra thick,gray,fill=white](0,0)circle(0.5cm);
\draw[very thick,blue] (-.3,0)--(.3,0);
\draw[very thick,blue,fill=blue](-.23,0)circle(0.125cm);
\draw[very thick,blue,fill=white](.23,0)circle(0.125cm);
\begin{scope}[xshift=1.5cm]
\draw[ultra thick,gray,fill=white](0,0)circle(0.5cm);
\draw[very thick,blue] (-.3,0)--(.3,0);
\draw[very thick,blue,fill=blue](-.23,0)circle(0.125cm);
\draw[very thick,blue,fill=white](.23,0)circle(0.125cm);
\end{scope}
\begin{scope}[xshift=2*1.5cm]
\draw[ultra thick,gray,fill=white](0,0)circle(0.5cm);
\draw[very thick,blue] (-.3,0)--(.3,0);
\draw[very thick,blue,fill=blue](-.23,0)circle(0.125cm);
\draw[very thick,blue,fill=white](.23,0)circle(0.125cm);
\end{scope}
\begin{scope}[xshift=3*1.5cm]
\draw[ultra thick,gray,fill=white](0,0)circle(0.5cm);
\draw[very thick,blue] (-.3,0)--(.3,0);
\draw[very thick,blue,fill=blue](-.23,0)circle(0.125cm);
\draw[very thick,blue,fill=white](.23,0)circle(0.125cm);
\end{scope}
\draw[very thick,-latex] (0.0,-.87)node[below]{$\iu \gammaii_1^+\gammaii_1^-$}--(0.0,-.21);
%%%%%%%%%%%%%%%%%%%%%%%%%%%%%%%%%%%%%%%%%%%%%%%%%%%%%%%%%%
% topological
\begin{scope}[xshift=8cm]
\draw (-0.5,1.2)node[right]{(a) faza topologiczna};

\draw[ultra thick,gray](0,.1)--(1.5,.1);
\draw[ultra thick,gray,xshift=1.5cm](0,.1)--(1.5,.1);
\draw[ultra thick,gray,xshift=2*1.5cm](0,.1)--(1.5,.1);

\begin{scope}[yshift=-.2cm]
\draw[ultra thick,gray](0,.1)--(1.5,.1);
\draw[ultra thick,gray,xshift=1.5cm](0,.1)--(1.5,.1);
\draw[ultra thick,gray,xshift=2*1.5cm](0,.1)--(1.5,.1);
\end{scope}

%balls
\draw[ultra thick,gray,fill=white](0,0)circle(0.5cm);
\draw[very thick,blue,fill=blue](-.23,0)circle(0.125cm);
\draw[very thick,blue,fill=white](.23,0)circle(0.125cm);
\begin{scope}[xshift=1.5cm]
\draw[ultra thick,gray,fill=white](0,0)circle(0.5cm);
\draw[very thick,blue,fill=blue](-.23,0)circle(0.125cm);
\draw[very thick,blue,fill=white](.23,0)circle(0.125cm);
\end{scope}
\begin{scope}[xshift=2*1.5cm]
\draw[ultra thick,gray,fill=white](0,0)circle(0.5cm);
\draw[very thick,blue,fill=blue](-.23,0)circle(0.125cm);
\draw[very thick,blue,fill=white](.23,0)circle(0.125cm);
\end{scope}
\begin{scope}[xshift=3*1.5cm]
\draw[ultra thick,gray,fill=white](0,0)circle(0.5cm);
\draw[very thick,blue,fill=blue](-.23,0)circle(0.125cm);
\draw[very thick,blue,fill=white](.23,0)circle(0.125cm);
\end{scope}
\draw[very thick,blue] (1.3,0)--(.35,0);
\draw[very thick,blue,xshift=1.5cm] (1.3,0)--(.35,0);
\draw[very thick,blue,xshift=2*1.5cm] (1.3,0)--(.35,0);
\draw[very thick,-latex] (0.75,-.87)node[below]{$\iu \gammaii_1^-\gammaii_2^+$}--(0.75,-.21);
\draw[very thick,-latex] (-.23,-.87)node[below]{$\gammaii_1^+$}--(-.23,-.21);
\draw[very thick,-latex,xshift=3*1.5cm] (.23,-.87)node[below]{$\gammaii_{\sites}^-$}--(.23,-.21);
\end{scope}

\end{tikzpicture}

    \caption[Ilustracja faz topologicznych w modelu Kitaeva]{Ilustracja faz topologicznych w modelu Kitaeva (a) faza trywialna (b) faza topologiczna.}
    \label{fig:mzmPhases}
\end{figure}

Wróćmy jeszcze do hamiltonianu \eqref{eq:kitaevHamiltonianTopoChain}.
Hamiltonian można rozwiązać wprowadzając nowe operatory fermionowe:
\begin{align}
    \widetilde\ai&=\tfrac12(\gammaii_{i+1}^+-\iu\gammai^-),\label{eq:fermionMode1}\\
    \aidt&=\tfrac12(\gammaii_{i+1}^++\iu\gammai^-),\label{eq:fermionMode2}
\end{align}
hamiltonian przyjmie wtedy następującą postać\footnote{Wyprowadzenie można znaleźć w dodatku \ref{chap:derivations}.}
\begin{equation}
    \hatH_{\text{topo}}^{\text{chain}} = -t^0\sum_{i=1}^{\sites-1}\left(1-2\nit\right) = -t^0\sum_{i=1}^{\sites-1} \widetilde\parity_i,\label{eq:kitaevHamiltonianTopoChainFermions}
\end{equation}
gdzie $\nit=\aidt\widetilde\ai$ i $\widetilde \parity_i = 1-2\nit$. 
Dla $t^0>0$ stan podstawowy tego hamiltonianu odpowiada sytuacji kiedy nie ma cząstek  w węzłach $i=1,\dots,\sites-1$, wszystkie $\nit$ dla podanego $i$ są nieobsadzone.
W hamiltonianie brakuje operatora $\nLt$, a więc niezależnie od tego czy stan $\nLt$ jest obsadzony czy nie, nie zmienia to energii układu (cząstka o energii $\Energy=0$).
Wniosek z tego jest następujący: układ posiada zdegenerowany stan podstawowy, jeden  z nieparzystą liczbą cząstek $\odd$ (ang. \textit{odd}) oraz drugi z parzystą liczbą cząstek $\even$ (ang. \textit{even}).
Oczywiście te stany układu są o innej parzystości:
\begin{align}
 \widetilde \parity_{\sites} \odd& =    \iu \gammaii_1^+\gammaii_{\sites}^- \odd = -\odd,\label{eq:parityKitaev1}\\
 \widetilde \parity_{\sites}\even& =    \iu \gammaii_1^+\gammaii_{\sites}^- \even = +\even.\label{eq:parityKitaev2}
\end{align}
Degeneracja stanu podstawowego jest warunkiem koniecznym\footnote{Nie jest to natomiast warunek wystarczający.} istnienia \MZM.
W wielu pracach autorzy wykorzystują degenerację jako podstawową przesłankę dotyczącą istnienia \MZM\ w zakresie danych parametrów.
\vspace{0.4cm}

Oczywiście, \MZM\ mogą istnieć w modelu Kitaeva poza  szczególnym przypadkiem opisanym w pkt. \ref{enum:topo}.
Rozważmy jednorodny hamiltonian \eqref{eq:kitaevHamiltonian} dla jednowymiarowego układu o otwartych warunkach brzegowych (łańcuch)\vspace{0.2cm}
\begin{equation}
    \hatH_{\text{Kitaev}}^{\text{chain}} = \sum_{i=1}^{\sites-1}
    \left[
    \left(\tuniform \, \aid \aii_{i+1} + \DeltaSCuniform\, \aid \aidi_{i+1}\right)
    + \hc\right] + \sum_{i=1}^{\sites} \muuniform\ni.
\end{equation}\vspace{0.2cm}
Dla takiego modelu, kiedy $|\DeltaSCuniform|>0$,
możemy wyróżnić dwie fazy: \textit{topologiczną} i \textit{trywialną}~\cite{kitaev.2001}.
Można pokazać analitycznie, że faza topologiczna jest obecna dla $|\muuniform|\le 2\tuniform$, a faza trywialna dla $|\muuniform|> 2\tuniform$~\cite{kitaev.2001}.
\MZM\ mogą się pojawić tylko w tej pierwszej fazie.
Należy tutaj podkreślić, że te warunki opisujące granice faz są spełnione jedynie w granicy termodynamicznej.
Dodatkowo w skończonych układach komutator~\eqref{eq:MZMcom} nie znika (poza szczególnym przypadkiem $\DeltaSC=|\tuniform|$).
Związane jest to bezpośrednio z rozkładem przestrzennym \MZM, a~bardziej precyzyjnie z ich przekrywaniem się.
\MZM\ zanikają wykładniczo od brzegów nanodrutu~\cite{kitaev.2001,kells.2015,katsura.schuricht.2015}.
\vspace{0.4cm}

Warto tutaj wspomnieć, że w przypadku  modelu Kitaeva rozszerzonego o oddziaływania wielociałowe, wyrażenie opisujące granicę tych dwóch faz jest nieco bardziej złożone~\cite{wieckowski.maska.2018,thomale.rachel.2013,katsura.schuricht.2015}.
Istnieje kilka metod badania obecności \MZM\ w układach z oddziaływaniami wielociałowymi, w tym kilka wskaźników sprawdzających istnienie faz topologicznych w takich układach~\cite{gergs.fritz.2016}. 
W tej rozprawie doktorskiej tematem przewodnim pracy są układy zawierające oddziaływania wielociałowe.
Z teoretycznego punktu widzenia, badanie kwantowych układów z oddziaływaniami wielociałowymi jest relatywnie trudnym zadaniem.
Istnieją wyrafinowane metody, dla przykładu \DMRG, które umożliwiają badanie układów zawierających tysiące węzłów, ale skuteczność tej metody jest ograniczona do badania krótko-zasięgowych oddziaływań.
Natomiast metoda \ED\  ograniczona jest tylko do relatywnie małych układów ($\sites\sim20$)~\cite{kozarzewski.mierzejewski.2019}, 
ale umożliwia badanie dowolnych oddziaływań, również tych dalekozasięgowych~\cite{wieckowski.ptok.2019}.

\ornament

%\section{Teoria grafów, a mody zerowe Majorany}
%
%Okazuje się, że dla modelu Kitaeva liczbę dostępnych \MZM\ w układzie można powiązać bezpośrednio ze stowarzyszonym z układem grafem skierowanym~\cite{future.pubication}.