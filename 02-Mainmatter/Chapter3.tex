\chapter{Topologiczne obliczenia kwantowe}\label{chap:topologicalQuantumComputing}

\section*{Opis rozdziału}

W tym rozdziale przedstawione zostaną najważniejsze cechy \MZM, które sprawiają, że są one tak interesującym tematem badawczym.
Zaprezentowane zostanie tutaj, w jaki sposób można wykorzystać \MZM\ do wykonywania obliczeń kwantowych i konstrukcji podstawowych bramek kwantowych.
Przedstawiona analiza w tym rozdziale została opracowana na podstawie prac
\cite{stanescu.2016,leijnse.flensberg.2012,sarma.freedman.2015,nayak.simon.2008,ivanov.2001,lahtinen.pachos.2017,field.simula.2018}.

\section{Statystyka nieabelowa i grupa warkoczowa}\label{sec:nonabelianStatistics}

W tym rozdziale skupiono się na rozważaniach bardziej ogólnych.
Założono, że w układzie znajduje się $2M$ \MZM: $\Gammaii_1,\dots,\Gammaii_{2M}$, które spełniają wszystkie założenia~\labelcref{eq:MZMherm,eq:MZMcom,eq:MZManticom} opisane w sekcji~\ref{sec:MZMdef}.
W identyczny sposób, jak dokonano to w sekcji~\ref{sec:kitaev} w równaniach \labelcref{eq:fermionMode1,eq:fermionMode2}, z \MZM\ można skonstruować $M$  modów fermionowych:
\begin{align}
    \widetilde\ai  &= \tfrac12(\Gammaii_{2i-1} - \iu\, \Gammaii_{2i}),\quad i=1,\dots,M\label{eq:anihilationMZM1}\\
    \aidt &= \tfrac12(\Gammaii_{2i-1} + \iu\, \Gammaii_{2i}),\label{eq:creationMZM2}
\end{align}
gdzie te fermiony spełniają standardowe relacje antykomutacji dla fermionów [równanie~\eqref{eq:aidtCommStandard}]:
\begin{equation}
\{ \widetilde\ai,\ajdt \} = \deltaij, \,\,\,\{\aidt,\ajdt\}=0.\label{eq:aidtComm} \end{equation}
W odróżnieniu od równań~\eqref{eq:fermionMode1},~\eqref{eq:fermionMode2}, gdzie tylko jeden fermion -- $\nLt$ -- miał zerową energię, każdy fermionowy mod utworzony z $\Gammai$ ma zerową energię. 
Z $2M$ \MZM\ można utworzyć $M$ modów fermionowych.
Każdy z tych fermionowych modów $\nit=\aidt\widetilde\ai$ może być pusty $\even$ lub zapełniony $\odd$ i może należeć odpowiednio do podprzestrzeni z parzystą lub nieparzystą liczbą cząstek.
Z tego wynika, że w układzie zawierającym $2M$ \MZM\ musi istnieć $2^M$ zdegenerowanych stanów podstawowych.
Te stany można ponumerować wartościami własnymi operatorów parzystości
\begin{equation}
    \widetilde\parity_i = 1- 2\nit = \iu\, \Gammaii_{2i-1}\Gammaii_{2i}.
\end{equation}
Taki operator parzystości, analogicznie jak w równaniach \eqref{eq:parityKitaev2} oraz \eqref{eq:parityKitaev2}, przyjmuje dwie wartości własne $\pm1$:
\begin{align}
    \widetilde\parity_i\odd_i &= -\odd_i,\label{eq:parityiodd}\\
    \widetilde\parity_i \even_i &= +\even_i,\label{eq:parityieven}
\end{align}
gdzie wprowadzona notacja $\even_i$ oznacza stan własny operatora parzystości $\widetilde\parity_i$, analogicznie jak w równaniach~\eqref{eq:parityKitaev1}, \eqref{eq:parityKitaev2}.
Takie operatory komutują ze sobą:
\begin{equation}
    [\widetilde \parity_i,\widetilde \parity_j] =0.
\end{equation}
Operatory fermionowe $\widetilde\ai$, $\aidt$ transformują stany podstawowe pomiędzy różnymi sektorami parzystości, posiadają analogiczne działanie jak operatory drabinkowe:\footnote{Dowód można znaleźć w~dodatku~\ref{chap:derivations}.}
\begin{align}
    \widetilde\ai\odd_i &= \even_i,\label{eq:aieven}\\[1ex]
    \widetilde\ai \even_i &= 0,\label{eq:aiodd}\\[1ex]
    \aidt \odd_i &= 0,\label{eq:aideven}\\[1ex]
    \aidt \even_i &= \odd_i.\label{eq:aidodd}
\end{align}
Z iloczynu operatorów parzystości można skonstruować całkowity operator parzystości w~układzie
\begin{equation}
    \widetilde\parity = \prod_{i=1}^M \widetilde \parity_i = \iu^M\prod_{j=1}^M \Gammaii_{2j-1}\Gammaii_{2j}
\end{equation}
Układy fermionowe, w stanie nadprzewodzącym lub w kontakcie z nadprzewodnikiem, chronione są symetrią parzystości, a  więc całkowita parzystość układu powinna być zachowana.
Rozważmy co się stanie z \MZM\ po adiabatycznej zamianie quasiczątek miejscami.
Dla danej pewnej fazy $\eta_i\in\{-1,+1\}$, \MZM\  zamieniają się pozycjami:
\begin{align}
    \Gammai &\to \Gammai'= \eta_{j} \Gammaii_{j},\\
    \Gammaj & \to \Gammaj' = \eta_{i} \Gammai.
\end{align}
Po takiej transformacji operator parzystości $\widetilde\parity$ powinien pozostać inwariantny
\begin{equation}
    \widetilde \parity \to \widetilde\parity' = 
    \iu^M \Gammaii_1 \cdots \Gammaii_i' \cdots \Gammaii_j' \cdots \Gammaii_{2M}=
    - \iu^M \Gammaii_1 \cdots \eta_i\Gammaii_i \cdots \eta_j\Gammaii_j \cdots \Gammaii_{2M} = \widetilde\parity\label{eq:parityCons}
\end{equation}
Aby równanie~\eqref{eq:parityCons} było spełnione, następujący warunek na fazy $\eta_i$ musi również być spełniony
\begin{equation}
    \eta_i\eta_j = -1.
\end{equation}
To równanie ma dwa rozwiązania:
\begin{align}
    \eta_i = \mp 1,\\
    \eta_j = \pm 1.
\end{align}
Pod tym rozwiązaniem ukryta jest najważniejsza własność \MZM\ --- nieabelowa statystyka (\acrshort{NAS}):
\begin{align}
    \Gammai &\to  \pm \Gammaii_{j},\\
    \Gammaj & \to \pm \Gammai.
\end{align}

Co się stanie ze stanem kwantowym $\state$ takiego układu po wymianie cząstek $\Gammai$, $\Gammaj$ (zakładając adiabatyczny proces wymiany)?
W mechanice kwantowej stan początkowy i stan końcowy  związany jest z pewną transformacją unitarną $\propagator$
\begin{equation}
\state \to \propagator\state.
\end{equation}
Rozwiązanie tego problemu można znaleźć zakładając, że w  układzie zachowana jest parzystość $\widetilde\parity$.
To implikuje fakt, że operator $\propagator$ komutuje z operatorem parzystości $\widetilde\parity$: 
\begin{equation}
    [\propagator,\widetilde \parity] = 0.
\end{equation}
Rozsądnym założeniem, jest również to, że operator $\propagator$ zależy tylko i wyłącznie od cząstek uczestniczących w wymianie $\propagator(\Gammai,\,\Gammaj)$.
Założono, postać takiego operatora
\begin{equation}
    \propagator(\Gammai,\,\Gammaj) = \exp[-\iu \mathcal A_{ij} \omega ],
\end{equation}
gdzie $\mathcal A_{ij}$ to operator hermitowski, a $\omega\in\realNumbers$.
Operator $\mathcal A_{ij}$ musi być funkcją operatorów $\Gammai,\,\Gammaj$.
Naturalnym kandydatem na postać $\mathcal A_{ij}$ jest operator parzystości zbudowany z \MZM\ uczestniczących w wymianie
\begin{equation}
    \mathcal A_{ij} = \iu \, \Gammai\, \Gammaj.
\end{equation}
Można wykazać, że postać $\propagator$ jest następująca%
 \footnote{\label{footnote:propagator}Wyprowadzenie można znaleźć w dodatku~\ref{chap:derivations}.}
\begin{equation}
    \propagator = \cos \omega + \Gammai\Gammaj\sin\omega.\label{eq:propagator1}
\end{equation}

\pagebreak

\noindent
Pozostało ustalić $\omega$. 
Można to zrobić pracując w obrazie Heisenberga~\cite{cohen-tannoudji.1977}%
%\footref{footnote:propagator}
\footnote{Wyprowadzenie można znaleźć w dodatku~\ref{chap:derivations}.}
\begin{equation}
    \Gammai \to \Gammai'= \propagator^\dagger\Gammai\propagator = \Gammai \cos(2\omega) + \Gammaj \sin(2\omega) = \pm\Gammaj,\label{eq:propagator2}
\end{equation}
co prowadzi do układu równań:
\begin{equation}
    \begin{cases}
    \cos(2\omega) &= 0,\\
    \sin(2\omega) &= \pm1.
    \end{cases}
\end{equation}
Ostatecznie są dwa rozwiązania dla $\omega\in[-\tfrac{\pi}2,\tfrac{\pi}2]$
\begin{equation}
    \omega = \pm \frac \pi4.
\end{equation}
Dwa rozwiązania $\omega$ nie są niczym nadzwyczajnym.
Znak można interpretować jako wymianę zgodną ze wskazówkami zegara, oraz wymianę przeciwną do wskazówek zegara.
Ostatecznie postać operatora $\propagator$ jest następująca
\begin{equation}
    \propagator(\Gammai,\,\Gammaj) = \exp[+\tfrac\pi4\Gammai\Gammaj] = 
    \tfrac1{\sqrt2}\left(1+\Gammai\Gammaj \right).
\end{equation}
Wybrano rozwiązanie $\omega =+\tfrac\pi4$ i przyjęto, że to rozwiązanie odpowiada wymianie zgodnej ze wskazówkami zegara.
Rozwiązanie $\propagator(\Gammaj,\,\Gammai)$ odpowiadające wymianie cząstek przeciwnie ze wskazówkami zegara odpowiada operacji odwrotnej do $\propagator(\Gammai,\,\Gammaj)$
\begin{equation}
    \propagator^{-1}(\Gammai,\,\Gammaj) = \propagator(\Gammaj,\,\Gammai) = \propagator^\dagger(\Gammai,\,\Gammaj) = \tfrac1{\sqrt2}(1-\Gammai\Gammaj).
\end{equation}

Dalej można zdefiniować nowy operator, $\braidOperator$,
 operator wyplatania
\begin{equation}
    \braidOperator = \propagator(\Gammai,\,\Gammaii_{i+1}) = \tfrac1{\sqrt2}(1+\Gammai \Gammaii_{i+1})\label{eq:braidOperator}.
\end{equation}
Zbiór takich operatorów $\{\braidOperatori_1,\dots,\braidOperatori_{M-1},\braidOperatori_1^{-1},\dots,\braidOperatori_{M-1}^{-1} \}$ stanowi zbiór generatorów \linebreak warkoczowej  $M$-włóknowej grupy Artina $\braidGroup $~\cite{adams.2004}.
W obrębie grupy warkoczowej $\braidGroup$ następujące równania są spełnione:
\begin{align}
    \forall_{|i-j|>1}\,\,\,\braidOperator\braidOperatori_j &= \braidOperatori_j\braidOperator,\label{eq:yangbaxter1}\\
    \forall_i\,\,\,\braidOperator\braidOperatori_{i+1}\braidOperator & = \braidOperatori_{i+1}\braidOperator\braidOperatori_{i+1}.\label{eq:yangbaxter2}
\end{align}
Wyżej wymienione równania stanowią tzw. \textit{równania Yanga--Baxtera}~\cite{yu.ge.2015,kauffman.2018,beenakker.2019}.
Dowód, że te równania są spełnione przez operator $\braidOperator$ zdefiniowany w \eqref{eq:braidOperator} został przeprowadzony w~dodatku~\ref{chap:derivations}.
Przykład w jaki sposób można reprezentować grupę warkoczową $\braidGroup$, włókna oraz operacje wyplatania $\braidOperator$ zilustrowano na rysunku~\ref{fig:braidGroupExample}.
\MZM\ mogą być interpretowane jako włókna w grupie $\braidGroup$.

\begin{figure}[ht!]
%\hspace{-3cm}
\begin{tikzpicture}[scale=0.9]

\begin{scope}[xshift=0cm]
\draw (0.5,0.5) node{(a)};
%\begin{tikzpicture}[scale=0.75]

\braid[number of strands=4,
style strands={1}{red,very thick},
style strands={2}{blue,very thick},
style strands={3}{green,very thick},
style strands={4}{orange,very thick}
](braid)
a_1
;
\node[at=(braid-1-s),above ] {$1$};
\node[at=(braid-2-s),above ] {$2$};
\node[at=(braid-3-s),above ] {$3$};
\node[at=(braid-4-s),above ] {$4$};

\braid[xshift=5cm,number of strands=4,
style strands={1}{red,very thick},
style strands={2}{blue,very thick},
style strands={3}{green,very thick},
style strands={4}{orange,very thick}
](braid)
a_1^{-1}
;
\node[at=(braid-1-s),above ] {$1$};
\node[at=(braid-2-s),above ] {$2$};
\node[at=(braid-3-s),above ] {$3$};
\node[at=(braid-4-s),above ] {$4$};

\draw (5,-0.75)node{$\neq$};

%\end{tikzpicture}
\end{scope}

\begin{scope}[yshift=-2cm]
\draw (0.5,0.) node{(b)};
%\begin{tikzpicture}[scale=0.75]

\braid[number of strands=4,
style strands={1}{red,very thick},
style strands={2}{blue,very thick},
style strands={3}{green,very thick},
style strands={4}{orange,very thick},
](braid)
a_1 a_1^{-1}
;
\node[at=(braid-1-s),above ] {\phantom{$1$}};
\node[at=(braid-2-s),above ] {};
\node[at=(braid-3-s),above ] {};
\node[at=(braid-4-s),above ] {};

\braid[xshift=5cm,number of strands=4,
style strands={1}{red,very thick},
style strands={2}{blue,very thick},
style strands={3}{green,very thick},
style strands={4}{orange,very thick},
style strands={5,6}{draw=none}
](braid)
a_5 a_5
;
\node[at=(braid-1-s),above ] {};
\node[at=(braid-2-s),above ] {};
\node[at=(braid-3-s),above ] {};
\node[at=(braid-4-s),above ] {};

\draw (5,-1.25)node{$=$};

%\end{tikzpicture}
\end{scope}

\begin{scope}[xshift=10cm,yshift=-0.cm]
\draw (0.5,0.5) node{(c)};
%\begin{tikzpicture}[scale=0.75]

\braid[number of strands=4,
style strands={1}{red,very thick},
style strands={2}{blue,very thick},
style strands={3}{green,very thick},
style strands={4}{orange,very thick},
](braid)
a_1 a_2 a_1
;
\node[at=(braid-1-s),above ] {$1$};
\node[at=(braid-2-s),above ] {$2$};
\node[at=(braid-3-s),above ] {$3$};
\node[at=(braid-4-s),above ] {$4$};

\braid[xshift=5cm,number of strands=4,
style strands={1}{red,very thick},
style strands={2}{blue,very thick},
style strands={3}{green,very thick},
style strands={4}{orange,very thick},
style strands={5,6}{draw=none}
](braid)
a_2 a_1 a_2
;
\node[at=(braid-1-s),above ] {$1$};
\node[at=(braid-2-s),above ] {$2$};
\node[at=(braid-3-s),above ] {$3$};
\node[at=(braid-4-s),above ] {$4$};

\draw (5,-1.75)node{$=$};

%\end{tikzpicture}
\end{scope}
\end{tikzpicture}

\caption[Ilustracja grupy warkoczowej]
{
Ilustracja grupy warkoczowej $\braidGroupi 4$, dla  $M=4$ włókien.
Włókna zostały ponumerowane.
Ilustracja równania:
(a) $\braidOperatori_1\neq \braidOperatori_1^{-1}$ (nieabelowość);
(b) $\braidOperatori_1 \braidOperatori_1^{-1}=\bbone$;
(c) Yanga--Baxtera \eqref{eq:yangbaxter2}. 
}
\label{fig:braidGroupExample}
\end{figure}







\ornament

\section{Bramki kwantowe}\label{sec:quantumGates}

W celu realizacji qubitu z wykorzystaniem \MZM\ potrzebne są co najmniej cztery \MZM\ $\Gammai$.
Spowodowane jest to symetrią parzystości w takich układach fermionowych, w których realizowane są \MZM.
W sekcji~\ref{sec:nonabelianStatistics} pokazano, że degeneracja stanu podstawowego zależy od liczby $\Gammai$ w układzie jak $2^{M/2}$, gdzie $M$ to liczba niezależnych \MZM\ $\Gammai$ w układzie.
Połowa tych zdegenerowanych stanów należy do sektora parzystego, a druga połowa do sektora nieparzystego przestrzeni Hilberta $\hilbertSpace$.
Posiadając w~układzie tylko dwie niezależne \MZM\ nie jesteśmy w stanie skonstruować bazy dla qubitu bazującego na \MZM.
Układ posiadający dwa niezależne $\Gammai$ posiada dwa stany podstawowe $\odd$ z sektora nieparzystego oraz $\even$ z sektora parzystego przestrzeni Hilberta $\hilbertSpace$.
Z uwagi na symetrię parzystości, przejścia pomiędzy stanami $\even\to\odd$ czy $\odd\to\even$ są niedozwolone.
Układ zawierający cztery \MZM: $\Gammaii_1$, $\Gammaii_2$, $\Gammaii_3$, $\Gammaii_4$ posiada cztery stany podstawowe: $\eeven=\even\,\kron\,\even$, $\eodd=\even\,\kron\,\odd$, $\oeven=\odd\,\kron\,\even$, $\oodd=\odd\,\kron\,\odd$.
Przyjęto następującą numerację quasicząstek
\begin{equation}
\oodd = \aidit{1}\aidit{2}\eeven.\label{eq:MZMstateEnum}
\end{equation}
Układ chroniony jest symetrią parzystości. 
Stany $\eeven$, $\oodd$ są parzyste, a stany $\oeven$, $\eodd$ są nieparzyste.
Przejścia, gdzie zachowana jest parzystość stanu kwantowego, w takim przypadku $\oodd\rightleftarrows\eeven$ oraz $\oeven\rightleftarrows\eodd$, są dozwolone.
Niedozwolone są natomiast przejścia pomiędzy stanami o różnej parzystości: $\oodd\rightleftarrows\eodd$,  $\oodd\rightleftarrows\oeven$, $\eeven\rightleftarrows\eodd$ oraz $\eeven\rightleftarrows\oeven$.
Grupa warkoczowa $\braidGroupi 4$ posiada $6$ generatorów: $\braidOperatori_1,\,\braidOperatori_2,\,\braidOperatori_3,\,\braidOperatori_1^\dagger,\,\braidOperatori_2^\dagger,\,\braidOperatori_3^\dagger$.
W dalszej części rozważań założono, że w układzie znajdowała się  parzysta liczba cząstek.
Stany bazowe qubitu $\qstate0$, $\qstate1$ można zdefiniować w następujący sposób:\footnote{Analogiczną analizę można przeprowadzić dla pozostałych stanów $\eodd$, $\oeven$.}
\begin{align}
    \qstate{0} &= \eeven,\label{eq:state0}\\
    \qstate1 &= \oodd.\label{eq:state1}
\end{align}
%\subsection{Bramka $Z$}
Następnie przeanalizowano co się stanie ze stanami $\qstate0$, $\qstate1$ po dokonaniu wyplatania $\braidOperatori_1$
\begin{equation}
    \qstate0 \to \braidOperatori_1 \qstate{0} = \tfrac1{\sqrt2}(1+\Gammaii_1\Gammaii_2)\qstate{0} =
    \tfrac1{\sqrt2}(1-\iu\widetilde\parity_1)\eeven =
    \tfrac1{\sqrt2}(1-\iu)\qstate0 = \exp(-\iu \tfrac\pi4)\qstate0.
\end{equation}
Oczywiście wyplatanie dokonane w odwrotnej kolejności prowadzi do przeciwnej zmiany fazy
\begin{equation}
    \qstate0 \to \braidOperatori_1^\dagger\qstate0 = \exp(+\iu\tfrac\pi4)\qstate0,
\end{equation}
co stanowi wspomnianą wcześniej właśność \MZM: nieabelowość.
Analogicznie stan $\qstate1$
\begin{equation}
    \qstate1 \to \braidOperatori_1\qstate1 = \exp(+\iu\tfrac\pi4)\qstate1.
\end{equation}
Podwójna wymiana $\braidOperatori_1^2$:
\begin{align}
    \qstate0 &\to \braidOperatori_1^2 \qstate0 = \exp(-\iu\tfrac\pi2)\qstate0 = -\iu \qstate0,\\
    \qstate1 &\to \braidOperatori_1^2 \qstate1 = \exp(+\iu\tfrac\pi2)\qstate1 = +\iu\qstate1.
\end{align}
Z dokładnością do fazy globalnej, operację $\braidOperatori_1^2$ można interpretować jako bramkę $\zgate$, którą w~bazie qubitu można zapisać jako%
\mymarginpar{\footnotesize
\begin{tikzpicture}[scale=0.7]
\braid[number of strands=4,
style strands={1}{red,very thick},
style strands={2}{blue,very thick},
style strands={3}{green,very thick},
style strands={4}{orange,very thick},
](braid)
a_1 a_1
;
\node[at=(braid-1-s),above ] {$\Gammaii_1$};
\node[at=(braid-2-s),above ] {$\Gammaii_2$};
\node[at=(braid-3-s),above ] {$\Gammaii_3$};
\node[at=(braid-4-s),above ] {$\Gammaii_4$};
\end{tikzpicture}
\captionof{figure}[Realizacja bramki $Z$.]{Realizacja bramki $\zgate$.}
\label{fig:zgate}
} 
\begin{equation}
    \zgate = \braidOperatori_1^2 = -\iu \paulii^z=-\iu
    \left[\begin{array}{lr}
    1     & 0 \\
    0     & -1
    \end{array}\right],
\end{equation}
gdzie $\qstate0=\left[ \begin{array}{c}
     1\\0
\end{array}\right]$, 
$\qstate0=\left[ \begin{array}{c}
     0\\1
\end{array}\right]$.
Ta dodatkowa faza $-\iu$ nie ma wpływu na wynik obliczeń kwantowych z wykorzystaniem takiej bramki $\zgate$.
Realizacja bramki została schematycznie  przedstawiona na rysunku~\ref{fig:zgate}.
%\subsection{Bramka $X$}
Następnie przeanalizowano działanie operatora $\braidOperatori_2$ na stany bazowe qubitu\footnote{\label{footnote:Xgate}%
Wyprowadzenie można znaleźć w dodatku~\ref{chap:derivations}.}
\begin{equation}
    \qstate0 \to \braidOperatori_2\qstate0 = \tfrac1{\sqrt2}(1+\Gammaii_2\Gammaii_3)\qstate0 = \tfrac1{\sqrt2}(\qstate0-\iu\qstate1).\label{eq:braidB2}
\end{equation}
Otrzymano superpozycję stanów bazowych.
Analogicznie stan $\qstate1$\footref{footnote:Xgate}
\begin{equation}
    \qstate 1 \to \braidOperatori_2\qstate 1 = \tfrac1{\sqrt2}(\qstate1 -\iu \qstate0).\label{eq:braidB21}
\end{equation}

\pagebreak

\noindent
Natomiast po podwójnej takiej operacji:\footnote{\label{footnote:X2gate}%
Wyprowadzenie można znaleźć w dodatku~\ref{chap:derivations}.}
\mymarginpar{\footnotesize
\begin{tikzpicture}[scale=0.7]
\braid[number of strands=4,
style strands={1}{red,very thick},
style strands={2}{blue,very thick},
style strands={3}{green,very thick},
style strands={4}{orange,very thick},
](braid)
a_2 a_2
;
\node[at=(braid-1-s),above ] {$\Gammaii_1$};
\node[at=(braid-2-s),above ] {$\Gammaii_2$};
\node[at=(braid-3-s),above ] {$\Gammaii_3$};
\node[at=(braid-4-s),above ] {$\Gammaii_4$};
\end{tikzpicture}
\captionof{figure}[Realizacja bramki $X$.]{Realizacja bramki $\xgate$.}
\label{fig:xgate}
} 
\begin{align}
\qstate0 &\to  \braidOperatori_2^2\qstate 0 = -\iu \qstate 1,\label{eq:braidB220}\\
\qstate1 &\to  \braidOperatori_2^2\qstate 1 = -\iu \qstate 0.\label{eq:braidB221}
\end{align}
 Co do nieistotnej fazy otrzymano bramkę $\xgate$ (co odpowiada klasycznej bramce \texttt{\normalsize not}) schematycznie przedstawioną na rysunku~\ref{fig:xgate}
 \begin{equation}
      \xgate = \braidOperatori_2^2 = -\iu \paulii^x=-\iu
    \left[\begin{array}{lr}
    0     & 1 \\
    1     & 0
    \end{array}\right].   
 \end{equation}
 %\subsection{Bramka $Y$}.
  Można pokazać, że następujące operacje prowadzą do realizacji bramki $\ygate$\footref{footnote:X2gate}  (rysunek~\ref{fig:ygate})%
\mymarginpar{\footnotesize
\begin{tikzpicture}[scale=0.7]
\braid[number of strands=4,
style strands={1}{red,very thick},
style strands={2}{blue,very thick},
style strands={3}{green,very thick},
style strands={4}{orange,very thick},
](braid)
a_1 ^{-1} a_2 a_2 a_1
;
\node[at=(braid-1-s),above ] {$\Gammaii_1$};
\node[at=(braid-2-s),above ] {$\Gammaii_2$};
\node[at=(braid-3-s),above ] {$\Gammaii_3$};
\node[at=(braid-4-s),above ] {$\Gammaii_4$};
\end{tikzpicture}
\captionof{figure}[Realizacja bramki $Y$.]{Realizacja bramki $\ygate$.}
\label{fig:ygate}
} 
  \begin{equation}
      \ygate = \braidOperatori_1\braidOperatori_2^2\braidOperatori_1^\dagger = -\iu\paulii^y =   -\iu\left[\begin{array}{cr}
    0     & -\iu \\
    \iu     & 0
    \end{array}\right]. \label{eq:braidYgate}
  \end{equation}
 % \subsection{Bramka Hadamarda}
W prosty sposób można zrealizować bramkę Hadamarda $\hadamard$\footref{footnote:X2gate} (rysunek~\ref{fig:hadamard})%
\mymarginpar{\footnotesize
\begin{tikzpicture}[scale=0.7]
\braid[number of strands=4,
style strands={1}{red,very thick},
style strands={2}{blue,very thick},
style strands={3}{green,very thick},
style strands={4}{orange,very thick},
](braid)
a_1  a_2 a_1
;
\node[at=(braid-1-s),above ] {$\Gammaii_1$};
\node[at=(braid-2-s),above ] {$\Gammaii_2$};
\node[at=(braid-3-s),above ] {$\Gammaii_3$};
\node[at=(braid-4-s),above ] {$\Gammaii_4$};
\end{tikzpicture}
\captionof{figure}[Realizacja bramki $\textsf{\textbf{\textit{H}}}$]{Realizacja bramki $\hadamard$.}
\label{fig:hadamard}
} 
\begin{equation}
    \hadamard = \braidOperatori_1 \braidOperatori_2 \braidOperatori_1 = -\iu\,\tfrac1{\sqrt2}
    \left[\begin{array}{lr}
    1     & 1 \\
    1     & -1
    \end{array}\right]. \label{eq:braidHgate}
\end{equation}
W dodatku~\ref{chap:MZMCNOT} czytelnik może znaleźć analizę układu \MZM, który może posłużyć do konstrukcji układu dwuqubitowego, wraz z realizacją bramki $\CNOT$.

\ornament

\section{Uniwersalność obliczeń, a bramka fazowa}\label{sec:universalQuantumComputing}

W przypadku komputerów klasycznych wystarczą dwie bramki  aby zagwarantować \textit{uniwersalność} obliczeń np. \texttt{OR}  i \texttt{NOT} lub \texttt{AND} i \texttt{NOT} lub jedna dwubitowa \texttt{NOR} lub \texttt{NAND}~\cite{nielsen.chuang.2011}.
Z tych ostatnich dwóch, w prosty sposób można zrealizować pierwsze cztery wymienione bramki.
Nieformalnie oznacza to, że wykorzystując wyżej wymienione klasyczne bramki można zrealizować dowolny klasyczny algorytm z dowolną dokładnością.
Zestaw takich bramek często nazywa się \textit{uniwersalnymi bramkami logicznymi}.

Bramka Hadamarda $\hadamard$, bramka fazowa%
\footnote{%
$
\phaseGatei{\tfrac\pi4}=
\begin{bmatrix}
    1     & 0 \\
    0     & \text \eee^{\iu  \frac \pi 4}
\end{bmatrix}
$}
$\phaseGatei{\tfrac\pi4}$ oraz bramka $\CNOT$%
\footnote{%
$
\CNOT=
\begin{bmatrix}
    1 & 0 & 0 & 0 \\
    0 & 1 & 0 & 0 \\
    0 & 0 & 0 & 1 \\
    0 & 0 & 1 & 0
\end{bmatrix}
$}
stanowią minimalny zestaw bramek, które gwarantują uniwersalność obliczeń kwantowych~\cite{nielsen.chuang.2011} --- \textit{uniwersalne}\linebreak \textit{bramki} \textit{kwantowe}.
Niestety okazuje się, że zestaw bramek, który można wygenerować za pomocą generatorów grupy warkoczowej $\braidGroup$, nie gwarantuje uniwersalności obliczeń~\cite{sarma.freedman.2015}.
Bramkę Hadamarda $\hadamard$ oraz bramkę $\CNOT$ można zrealizować z wykorzystaniem \MZM\ \linebreak z~zagwarantowaną ochroną topologiczną, co pokazano odpowiednio w sekcji~\ref{sec:quantumGates} oraz \linebreak w dodatku~\ref{chap:MZMCNOT}.
Wyzwanie stanowi bramka fazowa $\phaseGatei{\frac\pi4}$.
Tej ostatniej nie da się zrealizować wykorzystując \MZM\ z równoczesną zagwarantowaną ochroną topologiczną~\cite{sarma.freedman.2015}.
Da się natomiast skonstruować taką bramkę zbliżając i oddalając parę \MZM\ na pewien czas $\timeNormal$.
\MZM\ w układach zanikają wykładniczo od brzegów układu~\cite{kitaev.2001}.
W przypadku gdy \MZM\ przekrywają się na skutek oddziaływania, stany podstawowe ulegają rozszczepieniu o pewną energię $\deltaE=\Energy^{ee}-\Energy^{oo}$, gdzie odpowiednio energie $\Energy^{ee}$ oraz $\Energy^{oo}$ to energie własne stanów bazowych qubitu $\qstate0=\eeven$ oraz
 $\qstate 1=\oodd$  [równania \labelcref{eq:state0,eq:state1}].
Zakładając, że hamiltonian układu nie zależy od czasu i obecne jest niezerowe rozszczepienie energii $\deltaE\neq0$, po czasie $\timeNormal$ stany nabiorą odpowiednio fazy:
\begin{align}
\eeven &\to \eee^{-\iu \Energy^{ee} \timeNormal}\eeven,\\
\oodd &\to \eee^{-\iu \Energy^{oo} \timeNormal}\oodd.
\end{align}
Można to odpowiednio zaprezentować w postaci macierzy:
\begin{equation}
\phaseGate= 
\left[\begin{array}{cc}
\eee^{-\iu \Energy^{ee}\timeNormal} &  0\\
0 & \eee^{-\iu \Energy^{oo}\timeNormal}
\end{array}\right] = 
\eee^{-\iu \Energy^{ee}\timeNormal}\left[\begin{array}{cc}
1 &  0\\
0 & \eee^{\iu \deltaE\, \timeNormal}
\end{array}\right] =
\eee^{\iu\chi}\left[\begin{array}{cc}
1 &  0\\
0 & \eee^{\iu \theta}
\end{array}\right] 
\end{equation}
Z dokładnością do nieistotnej fazy globalnej $\chi=-\Energy^{ee}\timeNormal$ otrzymano bramkę fazową $\phaseGate$, gdzie $\theta = \deltaE\,\timeNormal$.
Zakładając idealną kontrolę nad układem, można by zrealizować bramkę $\phaseGatei{\tfrac\pi4}$ wymaganą do zagwarantowania uniwersalności obliczeń.
Oczywiście ciężko sobie wyobrazić taką perfekcyjną możliwość kontroli, w szczególności, że $\deltaE(\timeNormal)$ zależeć będzie od czasu dla takiego procesu zbliżania i oddalania \MZM.
Taka operacja z pewnością będzie wymagała dodatkowej korekcji błędów, np. z wykorzystaniem tzw. \textit{destylacji magicznych stanów}~\cite{bravyi.kitaev.2005,sarma.freedman.2015}.
Przedstawione powyżej rozwiązanie bazuje na fazie dynamicznej.
Zdecydowaną wadą takiego rozwiązania jest fakt, że faza $\theta$ zależy nie tylko od zmian parametrów hamiltonianu układu, ale również od czasu ewolucji.
W pracy~\cite{wieckowski.mierzejewski.2020}  zaproponowaliśmy inne podejście bazujące na fazie geometrycznej.
Prezentowane rozwiązanie również wymaga dodatkowej korekcji błędów po wykonaniu operacji z nim związanych.
Jednak takie rozwiązanie powinno generować mniejsze błędy niż standardowe podejście z wykorzystaniem fazy dynamicznej, dzięki wyeliminowaniu zależności od wspomnianej fazy.
  
\ornament