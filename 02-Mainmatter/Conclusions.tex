\chapter*{Podsumowanie}
\addcontentsline{toc}{part}{Podsumowanie}





\section*{Podsumowanie oraz wnioski}

W tej pracy przedstawiono wyniki dotyczące wpływu oddziaływań wielociałowych na własności zerowych modów Majorany w rozszerzonym modelu Kitaeva~\cite{kitaev.2001}.
Poza destruktywnym wpływem oddziaływań na czasy życia zerowych modów Majorany, oddziaływania mogą zmieniać zakres obszaru topologicznego ze względu na inne parametry modelu, np. ze względu na potencjał chemiczny.
Przed przedstawieniem oryginalnych wyników, w części~\hyperref[part:I]{\Romanbar{I}} oraz~\hyperref[part:II]{\Romanbar{II}} usystematyzowano potrzebną wiedzę teoretyczną wymaganą do zrozumienia omawianych wyników w części~\hyperref[part:III]{\Romanbar{III}}.
Poniżej podsumowano najważniejsze wyniki oraz wnioski z rozdziałów z części~\hyperref[part:IV]{\Romanbar{IV}}.

\subsection*{Rozdział~\ref{chap:identification}}

W rozdziale~\ref{chap:identification} przedstawiono działanie algorytmu (opisanego w Rozdziale~\ref{chap:LIOMs}) do znajdowania silnych (lub \textit{prawie} silnych), lokalnych, zerowych modów Majorany, na przykładzie modelu Kitaeva z oddziaływaniami wielociałowymi.
Stosowany algorytm jest ogólny i umożliwia badanie dowolnych hamiltonianów, w tym z dowolnymi oddziaływaniami wielociałowymi.
Zaprezentowany algorytm dobrze odtwarza standardowe wyniki znane w literaturze. 
Poza identyfikacją obecności \MZM, stosowany algorytm umożliwia badanie ich struktury przestrzennej.
Wprowadzone oddziaływania wielociałowe mogą zwiększać lokalizację \MZM\ na brzegach układu, co skutkuje zmniejszeniem ich wzajemnego przekrywania, co w~konsekwencji prowadzi do zwiększenia ich czasu życia.


\subsection*{Rozdział~\ref{chap:longrange}}

Podsumowując rozdział~\ref{chap:longrange}, dalekozasięgowe oddziaływania bardzo silnie wpływają na czasy życia \MZM. 
Co więcej, im dalszy zasięg tych oddziaływań, tym  występuje większy destrukcyjny efekt z nimi związany.
Takie oddziaływanie może mieć kluczowe znacznie z~punktu widzenia przyszłych materiałów, które będą wykorzystane do realizacji qubitów na bazie \MZM.
W materiałach takie oddziaływanie zanika wykładniczo wraz ze wzrostem odległości. 
W~celu zagwarantowania efektywnych obliczeń kwantowych  bazujących na \MZM, z gwarancją ochrony topologicznej, należy ograniczać dalekozasięgowe oddziaływania. 

\subsection*{Rozdział~\ref{chap:phaseGate}}



W rozdziale~\ref{chap:phaseGate} przedstawiono konstrukcję bramki fazowej, która jest ogólna i aplikowalna dla każdego modelu opisanego hamiltonianem z symetrią cząstka--dziura.
Kluczowym zjawiskiem, jakie jest wykorzystywane w tej bramce fazowej, jest przekrywanie się \MZM.
W celu realizacji bramki fazowej, protokół wyplatania \MZM\ należy zaaplikować na obu trójzłączach $\trijunction{12}$, $\trijunction{34}$.
Wykonanie protokołu na poszczególnych trójzłączach nie musi być zsynchronizowane w czasie, o ile czas trwania ewolucji na każdym złączu jest taki sam.
Na jednym ze złączy lokalny potencjał, umożliwiający transport \MZM, należy wybrać na przeciwny wzgledem lokalnego potencjału na drugim złączu, co w fizycznym eksperymencie może odpowiadać wybraniu przeciwnych napięć przyłożonych do złączy.
Na zrealizowanie takiego protokołu w prawdziwym eksperymencie czeka jeszcze wiele wyzwań i problemów.
Trójzłącza mogą nie być identyczne, bardziej rzeczywiste hamiltoniany mogą zawierać wyrazy łamiące symetrię cząstka--dziura, potencjały $\mui(\timeNormal)$ mogą mieć różne amplitudy na złączach.
Szacunkowy błąd zaprojektowanej bramki fazowej bazującej na fazie geometrycznej powinien być jednak mniejszy niż błąd bramki bazującej na fazie dynamicznej.
Przesunięcie fazowe bramki bazującej na fazie geometrycznej zależy tylko i wyłącznie od zmian parametrów układu, w przeciwieństwie do bramki bazującej na fazie dynamicznej, gdzie w tej drugiej przesunięcie fazowe zależy również od czasu trwania ewolucji.
Oczywiście, dla opisanego w pracy protokołu, bazującego na fazie geometrycznej, powinny być przeprowadzone dodatkowe korekty błędów~\cite{bravyi.kitaev.2005}.
Pokazano, że faza takiej bramki fazowej, bazującej na fazie geometrycznej, zależy od wszystkich parametrów modelu, a w szczególności od oddziaływań wielociałowych, co podkreśla jaką istotną rolę mają tego typu oddziaływania w niskowymiarowych układach kwantowych.


\ornament


\section*{Zrealizowane cele}

%%%%%Podsumowując, udało się zrealizować wszystkie cele postawione na początku rozprawy doktorskiej:
%%%%%\renewcommand{\labelitemii}{$\circ$}
%%%%%\begin{itemize}
%    \item usystematyzowano wiedzę dotyczącą \MZM:
%    \begin{itemize}
%        \item w części~\hyperref[part:I]{\Romanbar{I}} dokonując przeglądu literatury;
%        \item w części~\hyperref[part:II]{\Romanbar{II}} opisując podstawowy aparat matematyczny, model Kitaeva oraz wykorzystanie \MZM\ do realizacji podstawowych bramek kwantowych;
%    \end{itemize}
%%%%%    \item opracowano nową metodę umożliwiającą identyfikację silnych \MZM, w dowolnych układach oddziałujących fermionów
%%%%%    \begin{itemize}
%%%%%        \item algorytm omówiono w rozdziale~\ref{chap:LIOMs};
%%%%%        \item w rozdziałach~\ref{chap:identification} oraz \ref{chap:longrange} algorytm zaaplikowano do rozwiązania postawionego problemu;
%%%%%    \end{itemize}
%%%%%    \item zbadano wpływ oddziaływań wielociałowych na własności \MZM\ otrzymując oryginalne wyniki z tym związane opisane w części~\hyperref[part:III]{\Romanbar{III}};
%%%%%    \item opracowano nową implementację bramki fazowej dla qubitu bazującego na \MZM\ z~wykorzystaniem fazy geometrycznej co opisano w rozdziale~\ref{chap:phaseGate}.
%%%%%\end{itemize}

Podsumowując, udało się zrealizować wszystkie cele postawione na początku rozprawy doktorskiej.
Opracowano nową metodę umożliwiającą identyfikację silnych \MZM, w dowolnych układach oddziałujących fermionów.
Algorytm został szczegółowo omówiony w~rozdziale~\ref{chap:LIOMs}.
W rozdziałach~\ref{chap:identification} oraz \ref{chap:longrange} algorytm został zaaplikowany do rozwiązania postawionych problemów fizycznych.
Zbadany został wpływ oddziaływań wielociałowych na własności \MZM\ otrzymując oryginalne wyniki, które zostały opisane w części~\hyperref[part:III]{\Romanbar{III}}.
Opracowano również nową implementację bramki fazowej dla qubitu bazującego na \MZM\ z~wykorzystaniem fazy geometrycznej co opisano w rozdziale~\ref{chap:phaseGate}.

\ornament

\section*{Wyzwania}

Należy tutaj skomentować wyzwania oraz potencjalne perspektywy dotyczące wyników przedstawionych w niniejszej rozprawie doktorskiej.
Wszelkie badania zbliżają środowisko naukowe do realizacji pierwszego topologicznego komputera kwantowego. 
W~tej dziedzinie postęp naukowy  w~ostatnich latach jest spory, zarówno od strony teoretycznej, jak i doświadczalnej, jednak to są dopiero początki tych badań i  środowisko naukowe będzie czekać jeszcze wiele wyzwań i problemów do rozwiązania.

Zademonstrowany w pracy doktorskiej nowy algorytm umożliwia badanie silnych \MZM\ w układach z oddziaływaniami wielociałowymi.
Metoda ta bazuje na dokładnej diagonalizacji i niestety jej wykorzystanie jest ograniczone do badania stosunkowo małych układów (zawierających kilka węzłów).
Badanie złożonych układów kwantowych zawierających \MZM, które mogą posłużyć to przetwarzania kwantowej informacji, nie jest możliwe z wykorzystaniem tej nowej metody.

Realizacja przedstawionej w rozprawie doktorskiej nowej bramki fazowej bazującej na fazie geometrycznej dla qubitu bazującego na \MZM, powinna posiadać mniejszy błąd niż standardowa bramka fazowa bazująca na fazie dynamicznej.
Należy podkreślić, że przedstawiona nowa bramka fazowa, tak samo jak jej standardowa wersja bazująca na fazie dynamicznej, nie jest chroniona topologicznie.
Problemem realizacji nowej bramki fazowej mogą być jej warunki, jakie muszą być spełnione do jej poprawnego funkcjonowania.
Od strony technicznej konstrukcja dwóch identycznych, monoatomowych trójzłącz może być problematyczna.
Problematyczne może być również precyzyjne sterowanie pozycją \MZM\ w układzie tych trójzłącz.

Zasadniczym problemem wykorzystania \MZM\ do konstrukcji topologicznego komputera kwantowego, jest fakt, że dostępny zestaw topologicznie chronionych bramek nie gwarantuje uniwersalności obliczeń --- brakuje chronionej topologicznie bramki fazowej.
Znaczącym krokiem, mogłoby być poszukiwanie realizacji bardziej złożonych anyonów od \MZM, np. \textit{anyonów Fibbonaciego}~\cite{bonesteel.hormozi.2005,trebst.troyer.2008,nayak.simon.2008}.
W przeciwieństwie do \MZM, anyony Fibbonaciego mogą zapewnić zestaw topologicznie chronionych bramek gwarantujących uniwersalność obliczeń.


\ornament