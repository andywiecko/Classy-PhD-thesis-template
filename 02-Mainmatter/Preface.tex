\chapter*{Przedmowa}
\addcontentsline{toc}{part}{Przedmowa}



\section*{Struktura pracy}


Rozprawa doktorska została podzielona na cztery części:
\begin{itemize}
    \item W części \hyperref[part:I]{\Romanbar{I}} czytelnik zostaje wprowadzony w tematykę rozprawy --- \glslink{MZM}{zerowe mody Majorany (MZM, ang. Majorana zero modes)}.
    Przedstawiono motywację badań wykonanych z związku z przeprowadzoną rozprawą doktorską.
    Przedstawiony zostaje aparat matematyczny, własności operatorów, które pomagają zrozumieć ideę jaka stoi za wykonywaniem badań związanych z tym tematem.
    Zapoznano z podstawowymi operacjami, problemami oraz wyzwaniami jakie czekają przyszłych konstruktorów topologicznych komputerów kwantowych.
    Zapoznano z modelem Kitaeva, jako najprostszym modelem zawierającym \MZM.
    Przedstawiono w jaki sposób można wykorzystać \MZM\ do konstrukcji bramek kwantowych.
    
    \item Część \hyperref[part:II]{\Romanbar{II}} dotyczy metodyki badań naukowych.
    Zademonstrowano w jaki sposób buduje się elementy macierzowe hamiltonianów opisujących układ wielu cząstek, na przykładzie modelu Kitaeva.
    Dyskutowane jest wykorzystanie symetrii parzystości do ograniczenia przestrzeni Hilberta.
    Na podstawie opublikowanej pracy \cite{wieckowski.maska.2018}, wyprowadzono metody służące do identyfikacji \MZM\ w układach bez oddziaływań oraz z oddziaływaniami wielociałowymi.
    Przedstawiono stosowanie algorytmu do badania dynamiki w układach oddziałujących cząstek.
    
    \item W części \hyperref[part:III]{\Romanbar{III}} można znaleźć otrzymane wyniki wraz z ich analizą, które zostały opublikowane jako artykuły w czasopismach naukowych: 
    
    \begin{itemize}
        \item[\cite{wieckowski.maska.2018}] Andrzej Więckowski, Maciej M. Maśka, oraz Marcin Mierzejewski,  Identification  of Majorana Modes in Interacting Systems by Local Integrals of Motion, \href{http://dx.doi.org/10.1103/PhysRevLett.120.040504}{\textsw{Phys. Rev. Lett.}}, \href{http://dx.doi.org/10.1103/PhysRevLett.120.040504}{120:040504}, 2018;
        
        \item[\cite{wieckowski.ptok.2019}] Andrzej Więckowski oraz Andrzej Ptok,   Influence of long-range interaction on Majorana zero modes, \href{http://dx.doi.org/10.1103/PhysRevB.100.144510}{\textsw{Phys. Rev. B}}, \href{http://dx.doi.org/10.1103/PhysRevB.100.144510}{100:144510}, 2019;
        
        
        \item[\cite{wieckowski.mierzejewski.2020}] Andrzej Więckowski, Marcin Mierzejewski oraz Michał Kupczyński,
        Majorana phase gate based on the geometric phase,
        \href{http://dx.doi.org/10.1103/PhysRevB.101.014504}{\textsw{Phys. Rev. B}},
        \href{http://dx.doi.org/10.1103/PhysRevB.101.014504}{101:014504}, 2020.
        
    \end{itemize}
    
    
    \item Część \hyperref[part:IV]{\Romanbar{IV}} stanowi swoiste uzupełnienie poniższej rozprawy doktorskiej.
W tej części, poza informacją dotyczącą dorobku naukowego autora załączoną w dodatku~\ref{chap:cv}, czytelnik może znaleźć w niej wybrane dowody i wyprowadzenia (dodatek~\ref{chap:derivations}) oraz rozszerzone rozważania dotyczące dwuqubitowego układu z \MZM\ (dodatek~\ref{chap:MZMCNOT}).
Na końcu tej części zamieszczono spisy: akronimów, symboli, rysunków oraz bibliografię.
    
    
\end{itemize}




\ornament


\section*{Cele pracy}

Poniżej wymieniono cele niniejszej rozprawy doktorskiej.
\begin{itemize}

%    \item Systematyzacja wiedzy dotyczącej \MZM w układach bez oraz z oddziaływaniami wielociałowymi.
%    Systematyzacja dotyczy \MZM\ w ogólności, a także ich realizacji w rozszerzeniach modelu Kitaeva dla bezspinowych fermionów.

    \item Opracowanie nowej metody umożliwiającej identyfikację silnych \MZM, w dowolnych układach kwantowych.
    Metoda ta ma mieć zastosowanie do badań dowolnego hamiltonianu z oddziaływaniami wielociałowymi (lub bez).
    
    \item Badanie wpływu oddziaływań wielociałowych na statyczne i dynamiczne własności\linebreak \MZM.
    Sprawdzanie  obecności w układzie, rozkładu przestrzennego, stabilności\linebreak \MZM\ ze względu na parametry modelu.
    
    \item Opracowanie nowej bramki fazowej dla qubitu bazującego na \MZM\ z wykorzystaniem fazy geometrycznej.
    Sprawdzanie wpływu oddziaływań wielociałowych na proces wyplatania \MZM.
\end{itemize}

\ornament