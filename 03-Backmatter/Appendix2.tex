\chapter{Wybrane dowody i wyprowadzenia}
\label{chap:derivations}

W tym dodatku zamieszczono dowody i wyprowadzenia wybranych równań zaprezentowanych w rozprawie doktorskiej.

%\ornament

\section*{Równanie \eqref{eq:majoranaOrthogonality}}

\begin{flalign}
    (\gammai^\alpha | \gammaj^\beta )  &= \Tr (\gammai^\alpha \gammaj^\beta) / \Tr(\bbone) = \Tr\left(\tfrac12[\gammai^\alpha,\gammaj^\beta]+\tfrac12\{\gammai^\alpha,\gammaj^\beta\}\right)/\Tr(\bbone)=\\
    &= \tfrac12\Tr(\{\gammai^\alpha,\gammaj^\beta\})/\Tr(\bbone) = \tfrac12\Tr(2\deltaij\deltaab)/\Tr(\bbone) = \deltaij\deltaab&&
\end{flalign}

\ornament

\section*{Równanie \eqref{eq:kitaevHamiltonianTriv}}

Korzystając z równania \eqref{eq:parity}: $\ni = \tfrac12(1-\iu\gammai^+\gammai^-)$
\begin{flalign}
    \hatH_{\text{triv}} = \sum_i \mui \ni = \tfrac12\sum_i \mui\left(1-\iu\gammai^+\gammai^-\right).&&
\end{flalign}

\ornament

\section*{Równanie \eqref{eq:kitaevHamiltonianTopo}}
\begin{flalign}
\hatH_{\text{topo}} &= \sum_{\langle i,j\rangle} 
\left[
\left(\t0 \aid\aj + \t0 \aid\ajd 
\right) +\hc
\right]=\\
&= \sum_{\langle i,j\rangle}\frac{\t0}{4}
\left[
(\gammai^++\iu\gammai^-)(\gammaj^+-\iu\gammaj^-)
+
(\gammai^++\iu\gammai^-)(\gammaj^++\iu\gammaj^-)
\right]
+\hc=\\
&=
\sum_{\langle i,j\rangle}\frac{\t0}{4}
\Bigl(\gammai^+\gammaj^+-\iu\gammai^+\gammaj^-
+\iu\gammai^-\gammaj^++\gammai^-\gammaj^-
+\gammai^+\gammaj^++\iu\gammai^+\gammaj^-+\iu\gammai^-\gammaj^+-\gammai^-\gammaj^-\Bigr)+\hc=\\
&=
\sum_{\langle i,j\rangle}\frac{\t0}{4}
\Bigl(2\gammai^+\gammaj^++2\iu\gammai^-\gammaj^+\Bigr)+\hc = 
\iu\sum_{\langle i,j\rangle}\t0\,\gammai^-\gammaj^+
&&
\end{flalign}

\ornament

\section*{Równanie \eqref{eq:kitaevHamiltonianTopoChainFermions}}
\begin{flalign}
\nit &= \aidt\widetilde\ai = \tfrac14(\gammaii_{i+1}^++\iu\gammai^-)(\gammaii_{i+1}^+-\iu\gammai^-)=
\tfrac14 - \tfrac{\iu}4\gammaii_{i+1}^+\gammai^-
+ \tfrac{\iu}4\gammai^-\gammaii_{i+1}^+ + \tfrac14 = \\
&=\tfrac12 + \tfrac{\iu}2 \gammai^-\gammaii_{i+1}^+&&
\end{flalign}
\begin{flalign}
\hatH_{\text{topo}}^{\text{chain}} & = \iu\, t^0\sum_{i=1}^{\sites-1}\gammai^-\gammaii_{i+1}^+ = 
t^0\sum_{i=1}^{\sites-1}\left(2\nit -1\right) = - t^0\sum_{i=1}^{\sites-1}\widetilde\parity_i&&
\end{flalign}

\ornament

\section*{Równania \normalfont\labelcref{eq:aieven,eq:aiodd,eq:aideven,eq:aidodd}}
W dowodach skorzystano z równań \labelcref{eq:aidtComm,eq:parityiodd,eq:parityieven}.

\begin{flalign}
\widetilde\parity_i\even_i &= \even_i\\
(1-2\aidt\widetilde\ai)\even_i &= \even_i\\
\aidt\widetilde\ai\even_i &= 0\\
\event_i\aidt\widetilde\ai\even_i &= 0\\
\| \widetilde\ai \even_i \| &= 0\\
\widetilde\ai\even_i &= 0
\end{flalign}


\begin{flalign}
\widetilde\parity_i\odd_i & = -\odd_i\\
(1-2\aidt\widetilde\ai)\odd_i &= -\odd_i\\
(1-\aidt\widetilde\ai)\odd_i &= 0\\
\widetilde\ai\aidt\odd_i &= 0\\
\oddt_i\widetilde\ai\aidt\odd_i &= 0\\
\|\aidt\odd_i\| &= 0\\
\aidt\odd_i & = 0
\end{flalign}

\begin{flalign}
\widetilde\parity_i(\aidt \even_i) = (1-2\aidt\widetilde\ai)\aidt\even_i = \aidt \even_i - 2\aidt(1-\aidt\widetilde\ai)\even_i = -\aidt\even_i
\end{flalign}
Wektor $\aidt\even_i$ jest wektorem własnym operatora $\widetilde\parity_i$ z wartością własną $-1$, a zatem następująca relacja jest prawdziwa
\begin{equation}
\aidt\even_i = \odd_i.
\end{equation}

\begin{flalign}
\widetilde\parity_i(\widetilde\ai \odd_i) = 
(1-2\aidt\widetilde\ai)\widetilde\ai\odd_i = \
+\widetilde\ai\odd_i
\end{flalign}
Wektor $\widetilde\ai\odd_i$ jest wektorem własnym operatora $\widetilde\parity_i$ z wartością własną $+1$, a zatem następująca relacja jest prawdziwa
\begin{equation}
\widetilde\ai\odd_i = \even_i.
\end{equation}


\ornament 


\section*{Równanie \eqref{eq:propagator1}}

\begin{flalign}
\propagator & = \exp[\Gammai\Gammaj \omega] = \sum_{n=0}^\infty \tfrac1{n!}\omega^n (\Gammai\Gammaj)^n= \\
&=\sum_{n=0}^\infty \tfrac1{(2n)!}\omega^{2n} (\Gammai\Gammaj)^{2n}+
\sum_{n=0}^\infty \tfrac1{(2n+1)!}\omega^{2n+1} (\Gammai\Gammaj)^{2n+1}=\\
&=\sum_{n=0}^\infty \tfrac{(-1)^n}{(2n)!}\omega^{2n}+
\sum_{n=0}^\infty \tfrac{(-1)^n}{(2n+1)!}\omega^{2n+1} \Gammai\Gammaj=\cos\omega+\Gammai\Gammaj\sin\omega&&
\end{flalign}

\ornament

\section*{Równanie \eqref{eq:propagator2}}

\begin{flalign}
\propagator^\dagger \Gammai \propagator & = (\cos\omega -\Gammai \Gammaj \sin\omega)\Gammai(\cos\omega+\Gammai\Gammaj \sin\omega) = \\
&=\Gammai \cos^2\omega + \Gammaj \sin\omega\cos\omega + \Gammaj \sin\omega\cos\omega-\Gammai\sin^2\omega = \\
&= \Gammai \cos(2\omega) + \Gammaj \sin(2\omega) &&
\end{flalign}

\ornament


\section*{Równania \eqref{eq:yangbaxter1}, \eqref{eq:yangbaxter2}}


\begin{flalign}
\braidOperator\braidOperatori_j &= 
\tfrac1{\sqrt2}(1+\Gammai\Gammaii_{i+1})
\tfrac1{\sqrt2}(1+\Gammaj\Gammaii_{j+1})=\\
&= \tfrac12 (1+\Gammai\Gammaii_{i+1}+\Gammaj\Gammaii_{j+1}+\Gammai\Gamma_{i+1}\Gammaj\Gammaii_{j+1})\overset{|i-j|>1}{=}\\
&= \tfrac12(1+\Gammai\Gammaii_{i+1}+\Gammaj\Gammaii_{j+1}+\Gammaj\Gammaii_{j+1}\Gammai\Gamma_{i+1}) = \braidOperatori_j\braidOperator&&
\end{flalign}
\begin{flalign}
\braidOperator\braidOperatori_{i+1}\braidOperator & = 
\left[\tfrac1{\sqrt2}(1+\Gammai \Gammaii_{i+1})\right]
\left[\tfrac1{\sqrt2}(1+\Gammaii_{i+1} \Gammaii_{i+2})\right]
\left[\tfrac1{\sqrt2}(1+\Gammai \Gammaii_{i+1})\right]=\\
&=\tfrac1{2\sqrt2}\left[1+\Gammai\Gammaii_{i+1}+\Gammaii_{i+1}\Gammaii_{i+2}+ \Gammai\Gammaii_{i+2}\right]
\left[1+\Gammai\Gammaii_{i+1}\right]=\\
&=\tfrac1{2\sqrt2}[
1+\Gammai\Gammaii_{i+1}+\Gammaii_{i+1}\Gammaii_{i+2}+ \Gammai\Gammaii_{i+2}+\Gammai\Gammaii_{i+1}-1+\Gammaii_{i+2}\Gammai-\Gammaii_{i+2}\Gammaii_{i+1}]=\\
&=\tfrac1{2\sqrt2}[2\Gammai\Gammaii_{i+1}+2\Gammaii_{i+1}\Gammaii_{i+2}] =\tfrac1{\sqrt2}[\Gammai\Gammaii_{i+1}+\Gammaii_{i+1}\Gammaii_{i+2}]&&
\end{flalign}

    \begin{flalign}
\braidOperatori_{i+1}\braidOperator\braidOperatori_{i+1} & = 
\tfrac1{\sqrt2}\left[1+\Gammaii_{i+1} \Gammaii_{i+2}\right]
\tfrac1{\sqrt2}\left[1+\Gammai \Gammaii_{i+1}\right]
\tfrac1{\sqrt2}\left[1+\Gammaii_{i+1} \Gammaii_{i+2}\right]=\\
&=\tfrac1{2\sqrt2} \left[  1+\Gammai\Gammaii_{i+1}+\Gammaii_{i+1}\Gamma_{i+2}+\Gammaii_{i+2}\Gammaii_{i}\right]
[1+\Gammaii_{i+1} \Gammaii_{i+2}]=\\
&=\tfrac1{2\sqrt2} \left[  1+\Gammai\Gammaii_{i+1}+\Gammaii_{i+1}\Gamma_{i+2}+\Gammaii_{i+2}\Gammaii_{i}+
\Gammaii_{i+1}\Gammaii_{i+2}+
\Gammai\Gammaii_{i+2}-1+\Gammai\Gammaii_{i+1}
\right]
=\\
&=\tfrac1{2\sqrt2}[2\Gammai\Gammaii_{i+1}+2\Gammaii_{i+1}\Gammaii_{i+2}] = \tfrac1{\sqrt2}[\Gammai\Gammaii_{i+1}+\Gammaii_{i+1}\Gammaii_{i+2}] &&
\end{flalign}

\begin{equation}
\braidOperator\braidOperatori_{i+1}\braidOperator= 
\braidOperatori_{i+1}\braidOperator
\braidOperatori_{i+1}
\end{equation}

\ornament

\section*{Równania \eqref{eq:braidB2}, \eqref{eq:braidB21}}

Korzystając z równań 
\labelcref{eq:anihilationMZM1,eq:creationMZM2,eq:aidtComm,eq:aieven,eq:aiodd,eq:aideven,eq:aidodd,eq:MZMstateEnum}

\begin{flalign}
\braidOperatori_2\qstate0 &= \tfrac1{\sqrt2}(1+\Gammaii_2\Gammaii_3)\eeven =
\tfrac1{\sqrt2}[1+\iu(\widetilde\aii_1-\aidit{1})(\widetilde \aii_2 + \aidit{2})]\eeven=\\
	&=\tfrac1{\sqrt2}[\eeven-\iu \aidit{1}\aidit{2}\eeven] = \tfrac1{\sqrt2}[\eeven-\iu\oodd] = 
	\tfrac1{\sqrt2}[\qstate0-\iu\qstate1].
\end{flalign}

\begin{flalign}
\braidOperatori_2\qstate1 &= \tfrac1{\sqrt2}(1+\Gammaii_2\Gammaii_3)\oodd =
\tfrac1{\sqrt2}[1+\iu(\widetilde\aii_1-\aidit{1})(\widetilde \aii_2 + \aidit{2})]\oodd=\\
	&=\tfrac1{\sqrt2}[\oodd+\iu \widetilde\aii_{1}\widetilde\aii_{2}\oodd] =
	\tfrac1{\sqrt2}[\oodd+\iu \widetilde\aii_{1}\widetilde\aii_{2}\aidit{1}\aidit{2}\eeven ]=\\
	&=\tfrac1{\sqrt2}[\oodd-\iu \widetilde\aii_{1}\aidit{1}\widetilde\aii_{2}\aidit{2}\eeven ]
	=\tfrac1{\sqrt2}[\oodd-\iu (1-\aidit{1}\widetilde\aii_{1})(1-\aidit{2}\widetilde\aii_{2})\eeven ]=\\
	&=\tfrac1{\sqrt2}[\oodd-\iu\eeven] = 
	\tfrac1{\sqrt2}[\qstate1-\iu\qstate0]
\end{flalign}

\ornament

\section*{Równania \eqref{eq:braidB220}, \eqref{eq:braidB221}}


\begin{flalign}
\braidOperatori_2^2\qstate 0 &= \tfrac1{\sqrt2}\braidOperatori_2(\qstate0-\iu\qstate1) = \tfrac12 [\qstate0-\iu\qstate1-\iu(\qstate1 - \iu\qstate0)] = -\iu \qstate1&&
\end{flalign}
\begin{flalign}
\braidOperatori_2^2\qstate 1 = \tfrac1{\sqrt2} \braidOperatori_2 (\qstate1-\iu\qstate0) = \tfrac12[\qstate1 -\iu\qstate0 -\iu(\qstate0-\iu\qstate1)]=-\iu \qstate0&&
\end{flalign}


\ornament

\section*{Równania \eqref{eq:braidYgate}, \eqref{eq:braidHgate}}

\begin{flalign}
\braidOperatori_1\braidOperatori_2^2\braidOperatori_1^\dagger\qstate0 = \exp(+\iu\tfrac\pi4)
\braidOperatori_1\braidOperatori_2^2
\qstate0 = 
\exp(+\iu\tfrac\pi4)\braidOperatori_1(-\iu)\qstate1 = \qstate1. &&
\end{flalign}
\begin{flalign}
\braidOperatori_1\braidOperatori_2^2\braidOperatori_1^\dagger\qstate1 = \exp(-\iu\tfrac\pi4)\braidOperatori_1\braidOperatori_2^2
\qstate1 = 
\exp(-\iu\tfrac\pi4)\braidOperatori_1(-\iu)\qstate0 = -\qstate0. &&
\end{flalign}

\begin{equation}
\ygate = \braidOperatori_1\braidOperatori_2^2\braidOperatori_1^\dagger
\end{equation}

\begin{flalign}
\braidOperatori_1\braidOperatori_2 \braidOperatori_1\qstate 0 &=\exp(-\iu\tfrac\pi4) \braidOperatori_1^\dagger \braidOperatori_2  \qstate 0 =
\exp(-\iu\tfrac\pi4)\braidOperatori_1\tfrac1{\sqrt2}(\qstate 0 -\iu\qstate1) =\\ &=-\iu\tfrac1{\sqrt2}(\qstate0+\qstate1)&&
\end{flalign}

\begin{flalign}
\braidOperatori_1\braidOperatori_2 \braidOperatori_1\qstate 1 &=\exp(+\iu\tfrac\pi4) \braidOperatori_1 \braidOperatori_2  \qstate 1 =
\exp(+\iu\tfrac\pi4)\braidOperatori_1\tfrac1{\sqrt2}(\qstate 1 -\iu\qstate0) =\\
&=-\iu\tfrac1{\sqrt2}(\qstate0-\qstate1)&&
\end{flalign}

\begin{equation}
\hadamard = \braidOperatori_1\braidOperatori_2 \braidOperatori_1
\end{equation}

\ornament

\section*{Równanie \eqref{eq:gammaAvg}}

\begin{flalign}
    \barGammaii &=  \lim_{\timeNormal\to\infty}\frac1{\timeNormal}\int\limits_0^{\timeNormal}\text d\timeNormal' \Gammaii(\timeNormal')=
    \lim_{\timeNormal\to\infty}\frac1{\timeNormal}\int\limits_0^{\timeNormal}\text d\timeNormal' \eee^{+\iu \hatH \timeNormal'} \Gammaii \eee^{-\iu \hatH \timeNormal'}=\\
    &=
    \lim_{\timeNormal\to\infty}\frac1{\timeNormal}\int\limits_0^{\timeNormal}\text d\timeNormal' \eee^{+\iu \hatH \timeNormal'} \sum_{nm} \qstatet n \Gammaii \qstate m \, \qstate n \qstatet m \eee^{-\iu \hatH \timeNormal'}=\\
        &=
    \lim_{\timeNormal\to\infty}\frac1{\timeNormal}\sum_{nm}\int\limits_0^{\timeNormal}\text d\timeNormal' \eee^{-\iu (\Energy_m-\Energy_n) \timeNormal'}  \Gammaii_{nm} \, \qstate n \qstatet m =\\
    &=\lim_{\timeNormal\to\infty}\sum_{nm}\thetaH{\tfrac1{\timeNormal}-|\Energy_n-\Energy_m|}\,\,\Gammaii_{nm}\,\,\,\qstate n\qstatet m=\label{eq:mzmAvgTheta}\\
    &= \sum_{\Energy_n=\Energy_m}\,\Gammaii_{nm} \,\,\,\qstate n\qstatet m.&&
\end{flalign}

\ornament

\section*{Równanie \eqref{eq:normOpt}}
    
Korzystając z własności liniowości iloczynu skalarnego oraz z tożsamości \eqref{eq:barGammabarGamma}
\begin{flalign}
 \|(\Gammaii-\barGammaii) \|^2 &=
 (\Gammaii-\barGammaii|\Gammaii-\barGammaii) = (\Gammaii|\Gammaii) - (\Gammaii|\barGammaii) - (\barGammaii|\Gammaii) + (\barGammaii|\barGammaii)=\\
 &=1 - (\barGammaii|\barGammaii) = 1 - \| \barGammaii \|^2 .&&
\end{flalign}

\ornament

\section*{Równanie \eqref{eq:barGammabarGamma}}

Ostatnia nierówność jest oczywista i wynika z własności przemienności iloczynu skalarnego
\begin{flalign}
    (\barGammaii | \Gammaii) = (\Gammaii|\barGammaii).
\end{flalign}
Następnie udowodniono ostatnią nierówność bez zakładania granicy $\lim_{\timeNormal\to\infty}$ --- dla dowolnych czasów $\timeNormal$ (jako $Z$ oznaczono $Z=\Tr\,\,\bbone$):
\begin{flalign}
    (\barGammaii|\barGammaii) &= \tfrac1Z\Tr(\barGammaii\barGammaii)=\\
    &=
    \tfrac1Z\sum_n\qstatet n\left( \sum_{n'm'}\thetaH{\tfrac1{\timeNormal}-|\Energy_{n'}-\Energy_{m'}|}\Gammaii_{n'm'}\qstate {n'} \qstatet {m'}
    \sum_{k'l'}\thetaH{\tfrac1{\timeNormal}-|\Energy_{k'}-\Energy_{l'}|}\Gammaii_{k'l'}\qstate {k'} \qstatet {l'}\right)\qstate n=\\
    &=
    \tfrac1Z\sum_{nn'm'k'l'}
    \thetaH{\tfrac1{\timeNormal}-|\Energy_{n'}-\Energy_{m'}|}
    \thetaH{\tfrac1{\timeNormal}-|\Energy_{k'}-\Energy_{l'}|}
    \Gammaii_{n'm'}
    \Gammaii_{k'l'}
    \deltaii{nn'}
    \deltaii{m'k'}
    \deltaii{l'n}=\\
    &=
    \tfrac1Z\sum_{n'm'}
    \thetaH{\tfrac1{\timeNormal}-|\Energy_{n'}-\Energy_{m'}|}
    \thetaH{\tfrac1{\timeNormal}-|\Energy_{m'}-\Energy_{n'}|}
    \Gammaii_{n'm'}
    \Gammaii_{m'n'}=\\
    &=
    \tfrac1Z\sum_{n'm'}
    \thetaH{\tfrac1{\timeNormal}-|\Energy_{n'}-\Energy_{m'}|}
    \Gammaii_{n'm'}
    \Gammaii_{m'n'}=(\barGammaii|\Gammaii),
\end{flalign}
gdzie skorzystano tutaj z równania~\eqref{eq:mzmAvgTheta} oraz z własności funkcji Heaviside'a
\begin{equation}
    \thetaHs{x} = \thetaH{x}.
\end{equation}
\ornament






\section*{Równanie \eqref{eq:timeTauAvgDef}}

\begin{flalign}
\thickbar \Gammaii^{\timeTau} &= \int\limits_{-\infty}^\infty\text d\timeNormal\,\, \Gammaii(\timeNormal) \tfrac{\sin(\timeNormal/\timeTau)}{\pi \timeNormal}=
\int\limits_{-\infty}^\infty\text d\timeNormal\,\, \eee^{+\iu\hatH \timeNormal}\sum_{nm}\Gammaii_{nm}\qstate n\qstatet m \eee^{-\iu\hatH \timeNormal} \tfrac{\sin(\timeNormal/\timeTau)}{\pi \timeNormal}=
\\
&=
\sum_{nm}\Gammaii_{nm}\qstate n\qstatet m \int\limits_{-\infty}^\infty\text d\timeNormal\,\, \eee^{-\iu(\Energy_m-\Energy_n) \timeNormal} \tfrac{\sin(\timeNormal/\timeTau)}{\pi \timeNormal}=
\\
  & 
 = \sum_{nm}\thetaH{\tfrac{1}{\timeTau}-|\Energy_n-\Energy_m|} \Gammaii_{nm} \qstate n \qstatet m
  = \sum_{|\Energy_n-\Energy_m|<\tfrac 1{\timeTau}} \Gammaii_{nm} \qstate n \qstatet m &&
\end{flalign}

\ornament

\section*{Równanie \eqref{eq:standardCorr}}


\begin{flalign}
    (\Gammaii(t)|\Gammaii) &= \Tr(\Gammaii(t)|\Gammaii)/\Tr(\bbone)=
    \tfrac1Z \sum_{n'} \qstatet {n'}\left(\eee^{+\iu \hatH \timeNormal}\Gammaii \eee^{-\iu \hatH \timeNormal}
    \Gammaii\right)\qstate {n'}=\\
    &=
 \tfrac1Z \sum_{nm}\eee^{\iu (\Energy_m-\Energy_n) \timeNormal}\Gammaii_{nm}
    \Gammaii_{mn}=
        \tfrac1Z\sum_{nm}\eee^{\iu(\Energy_m-\Energy_n)\timeNormal}|\qstatet n\Gammaii \qstate m|^2&&
\end{flalign}
gdzie $Z=\Tr\,\,\bbone$.

\ornament

\section*{Równanie \eqref{eq:tautimeRelationship}}


\begin{flalign}
    (\barGammaii^{\timeTau}|\Gammaii ) &= \tfrac1Z\sum_{nm}\thetaH{\tfrac1{\timeTau}-|\Energy_n-\Energy_m|} |\qstatet n\Gammaii\qstate m|^2=\\
    &=
    \tfrac1Z\sum_{nm} |\qstatet n\Gammaii\qstate m|^2\int\limits_{-\tfrac1{\timeTau}}^{\tfrac1{\timeTau}}
    \text d\omega \,\, \diracDelta{\omega+\Energy_m-\Energy_n}=
    \\
    &=
    \tfrac1Z\sum_{nm} |\qstatet n\Gammaii\qstate m|^2\int\limits_{-\tfrac1{\timeTau}}^{\tfrac1{\timeTau}}
    \text d\omega \,\, \tfrac1{2\pi}\int\limits_{-\infty}^{\infty}\text d\timeNormal\,\, \eee^{\iu(\omega+\Energy_m-\Energy_n)\timeNormal}=\\
    &=
    \frac1{2\pi}
\int\limits_{-\frac1{\timeTau}}^{\frac1{\timeTau}}\text d\omega \,\, \int\limits_{-\infty}^\infty \text d\timeNormal\,\, \eee^{\iu\omega \timeNormal} (\Gammaii(\timeNormal)|\Gammaii)&&
\end{flalign}
\ornament


\section*{Równanie \labelcref{%
eq:barGammaGammaVcorr,%
eq:GammaGammaFourier,%
eq:barGammaGammaFullcorr}}

Przeprowadzenie dowodu rozpoczęto od policzenia transformaty Fouriera:
\begin{flalign}
    (\Gammaii(\omega)|\Gammaii)_I &= \int\limits_{-\infty}^{\infty}\text d\timeNormal\,\, \eee^{\iu\omega\timeNormal} \eee^{-|\timeNormal|/\timeTauI} = 
    \int\limits_{-\infty}^0\text d\timeNormal\,\,\eee^{\iu\omega\timeNormal}\eee^{\timeNormal/\timeTauI}+
    \int\limits_{0}^\infty\text d\timeNormal\,\,\eee^{\iu\omega\timeNormal}\eee^{-\timeNormal/\timeTauI}=\\
    &=
    \frac{1}{\iu\omega+\tfrac1{\timeTauI}}\eee^{(\iu\omega+\tfrac{1}{\timeTauI})\timeNormal}\biggr|_{-\infty}^0
    +
    \frac{1}{\iu\omega-\tfrac1{\timeTauI}}\eee^{(\iu\omega-\tfrac{1}{\timeTauI})\timeNormal}\biggr|_{-\infty}^0
    =
    \frac{2\timeTauI}{1+(\omega\timeTauI)^2}.
    &&
\end{flalign}
Z wykorzystaniem tej transformaty Fouriera wykonując proste obliczenia, można pokazać korelację związaną z mechanizmem relaksacji od oddziaływań wielociałowcych:
\begin{flalign}
    (\barGammaii^{\timeTau}|\Gammaii )_I &= 
    \frac1{2\pi}
\int\limits_{-\frac1{\timeTau}}^{\frac1{\timeTau}}\text d\omega \,\, \int\limits_{-\infty}^\infty \text d\timeNormal\,\, \eee^{\iu\omega \timeNormal} (\Gammaii(\timeNormal)|\Gammaii)_I = 
 \frac1{2\pi}
\int\limits_{-\frac1{\timeTau}}^{\frac1{\timeTau}}\text d\omega \,\, (\Gammaii(\omega)|\Gammaii)_I=\\
&=  \frac1{\pi}
\int\limits_{-\frac1{\timeTau}}^{\frac1{\timeTau}}\text d\omega \,\, \frac{\timeTauI}{1+(\omega\timeTauI)^2} = 
\tfrac1{\pi}\arctan\left(\tfrac{\timeTauI}{\timeTau}\right) - 
\tfrac1{\pi}\arctan\left(-\tfrac{\timeTauI}{\timeTau}\right) = \tfrac2{\pi}\arctan\left(\tfrac{\timeTauI}{\timeTau}\right).
\end{flalign}
Analogicznie można obliczyć pełną korelację: uwzględniającą wpływ oddziaływań wielociałowych oraz relaksacji oddziaływania modów Majorany:
\begin{flalign}
        (\barGammaii^{\timeTau}|\Gammaii)_{IM} &=  \frac1{2\pi}
\int\limits_{-\frac1{\timeTau}}^{\frac1{\timeTau}}\text d\omega \,\, \int\limits_{-\infty}^\infty \text d\timeNormal\,\, \eee^{\iu\omega \timeNormal} (\Gammaii(\timeNormal)|\Gammaii)_{IM} = 
 \frac1{2\pi}
\int\limits_{-\frac1{\timeTau}}^{\frac1{\timeTau}}\text d\omega \,\, (\Gammaii(\omega-\tfrac1{\timeTauM})|\Gammaii)_I= \\
        &= \frac1{\pi}
\int\limits_{-\frac1{\timeTau}}^{\frac1{\timeTau}}\text d\omega \,\, \frac{\timeTauI}{1+[(\omega-\tfrac{1}{\timeTauM})\timeTauI]^2} = \\
        &=
    \frac1\pi\left
    \{
    \arctan\left[\left(\frac1{\timeTau}-\frac1{\timeTauM}\right)
    \timeTauI
    \right]
    +
    \arctan\left[\left(\frac1{\timeTau}+\frac1{\timeTauM}\right)
    \timeTauI
    \right]
    \right\}.
\end{flalign}

\ornament

\section*{Równanie~\eqref{eq:gammaPlusGammaMinusOrthogonality}}

Korzystając z założenia, że $\hatH_{ij}\in\realNumbers$ i postaci operatorów $\gammai^\pm$ [równania~\labelcref{eq:operatorMajoranaPlus,eq:operatorMajoranaMinus}]:
\begin{align}
    (\bargammai^+|\bargammaj^-)&=(\barai{i}+\baradi{i}|\iu(\barai{j}-\baradi{j})) = 
    \iu\left( 
    (\barai{i}|\barai{j})-(\barai{i}|\baradi{j})+(\baradi{i}|\barai{j})-(\baradi{i}|\baradi{j})
    \right)=\\
    &=
    \iu\left( 
    (\barai{i}|\barai{j})-(\barai{i}|\barai{j})^*+(\baradi{i}|\barai{j})-(\baradi{i}|\barai{j})^*
    \right)=0.
\end{align}


\ornament


\section*{Równanie~\eqref{eq:parityMappingCondition}}

\begin{align}
    1 &= \qstatet n \barGammaii^2 \qstate n = \qstatet n\biggl(
    \sum_{n',m':\Energy_{n'}=\Energy_{m'}}\Gammaii_{n'm'} \qstate {n'} \qstatet{m'}
    \biggr)
    \biggl(
    \sum_{n'',m'':\Energy_{n''}=\Energy_{m''}}\Gammaii_{n''m''} \qstate {n''} \qstatet{m''}
    \biggr)
    \qstate n=\\
    &=
    \sum_{n',m':\Energy_{n'}=\Energy_{m'}}\,
    \sum_{n'',m'':\Energy_{n''}=\Energy_{m''}}
    \Gammaii_{n'm'}\Gammaii_{n''m''}
    \deltaii{nn'}
    \deltaii{m'n''}
    \deltaii{m''n}=\\
    &=\sum_{m:\,\Energy_{n}=\Energy_{m}}\,\Gammaii_{nm}\Gammaii_{mn}
    =\sum_{m:\,\Energy_{n}=\Energy_{m}}\,|\Gammaii_{nm}|^2
\end{align}

\ornament



\section*{Równania~\normalfont\labelcref{eq:shiba1,eq:shiba2,eq:shiba3}}

Skorzystano z postaci transformacji $\shiba$ z równania~\eqref{eq:shiba4}
\begin{equation}
\shiba = (\iu \gammaii_{\sites}^-)  \gammaii_{\sites-1}^+ \cdots \gammaii_2^+ (\iu \gammaii_1^-)
\end{equation}
i równania sprzężonego
\begin{equation}
\shibad = 
(-\iu \gammaii_1^-) \gammaii_2^+ \cdots \gammaii_{\sites-1}^+  (-\iu\gammaii_{\sites}^-)  
\end{equation}
do pokazania, że jest to transformacja unitarna:
\begin{flalign}
\shiba \shibad &=         
(\iu \gammaii_{\sites}^-)  \gammaii_{\sites-1}^+ \cdots \gammaii_2^+ (\iu \gammaii_1^-)
(-\iu \gammaii_1^-) \gammaii_2^+ \cdots \gammaii_{\sites-1}^+  (-\iu\gammaii_{\sites}^-)  = \bbone\\
\shibad \shiba &=        \bbone ,&&
\end{flalign}
gdzie skorzystano tutaj z własności $(\gammai^\alpha)^2=\bbone$.
Do transformacji operatorów kreacji i anihilacji warto skorzystać z ich postaci w bazie operatorów Majorany
$\ai = \tfrac12(\gammai^+-\iu\gammai^-)$, $\aid =\tfrac12(\gammai^++\iu\gammai^-)$.
Poniżej pokazano, w jaki sposób transformują się operatory Majorany.
\begin{flalign}
\shiba \gammai^+ \shibad &= 
(\iu \gammaii_{\sites}^-)  \gammaii_{\sites-1}^+ \cdots \gammaii_2^+ (\iu \gammaii_1^-)
\gammai^+
(-\iu \gammaii_1^-) \gammaii_2^+ \cdots \gammaii_{\sites-1}^+  (-\iu\gammaii_{\sites}^-)=\\
&=
\begin{cases}
(-1)^{\sites}=-1,&i\,\, \text{nieparzyste}\\
(-1)^{i-1}(-1)^{\sites-i}=1,&i\,\, \text{parzyste}\\
\end{cases}\,\,=(-1)^i \gammai^+,&&
\end{flalign}
gdzie skorzystano tutaj z założenia, że $\sites$ jest nieparzyste oraz z reguł antykomutacji operatorów Majorany $\{\gammai^\alpha,\gammaj^\beta\}=2\deltaij\deltaab$.
Analogicznie można pokazać transformację operatora $\gammai^-$:
\begin{flalign}
\shiba \gammai^- \shibad &= 
(\iu \gammaii_{\sites}^-)  \gammaii_{\sites-1}^+ \cdots \gammaii_2^+ (\iu \gammaii_1^-)
\gammai^-
(-\iu \gammaii_1^-) \gammaii_2^+ \cdots \gammaii_{\sites-1}^+  (-\iu\gammaii_{\sites}^-)=\\
&=
\begin{cases}
(-1)^{\sites-i}(-1)^{i-1}=1,&i\,\, \text{nieparzyste}\\
(-1)^{\sites}(-1)^{\sites-i}=-1,&i\,\, \text{parzyste}\\
\end{cases}\,\,=(-1)^i (-1)\gammai^-.&&
\end{flalign}
Ostatecznie:
\begin{flalign}
\shiba\ai\shibad = \shiba \tfrac12(\gammai^+-\iu\gammai^-)\shibad=
(-1)^i\tfrac12( \gammai^++\iu\gammai^-) = (-1)^i \aid,&&\label{eq:aishibaderivation}
\end{flalign}
\begin{flalign}
\shiba\aid\shibad = \shiba \tfrac12(\gammai^++\iu\gammai^-)\shibad=
(-1)^i\tfrac12( \gammai^+-\iu\gammai^-) = (-1)^i \ai.&&\label{eq:aidshibaderivation}
\end{flalign}


\ornament



\section*{Równanie~\eqref{eq:shiba5}}
W celu pokazania, że hamiltoniany transformują się tak jak w równaniu~\eqref{eq:shiba5}, należy pokazać, w jaki sposób transformują się wszystkie wyrazy hamiltonianów:
\begin{itemize}
\item  część związana z przeskokiem cząstek
\begin{flalign}
\shiba\aid\aj\shibad = \shiba\aid\shibad\shiba\aj\shibad = (-1)^{i+j}\ai\ajd = (-1)\ai\ajd=\ajd\ai = (\aid\aj)^\dagger,&&
\end{flalign}
\item część związana z kreacją cząstek
\begin{flalign}
\shiba\aid\ajd\shibad = \shiba \aid \shibad \shiba \ajd \shibad = (-1)^{i+j}\ai \aj = (-1)\ai\aj = \aj\ai = (\aid\ajd)^\dagger,&&
\end{flalign}
\item część związana z potencjałem jednocząstkowym
\begin{flalign}
\shiba\nz{i}\shibad=\shiba\aid\shibad\shiba\ai\shibad-\tfrac12=(-1)^{2i}\ai\aid-\tfrac12=1-\aid\ai\tfrac12=-\nz i,&&
\end{flalign}
\item część związana z potencjałem dwucząstkowym
\begin{flalign}
\shiba\nz i  \nz j \shibad=\shiba\nz i\shibad \shiba \nz j \shibad = (-\nz i)(-\nz j) = \nz i\nz j,&&
\end{flalign}
\end{itemize}
gdzie $i\neq j$ i zgodnie z założeniami nieparzystego $\sites$, układ tworzy dwudzielną sieć, dla której $i+j$ jest liczbą nieparzystą i skorzystano z relacji antykomutacji operatorów fermionowych $\ai\aid+\aid\ai=1$, unitarności $\shibad\shiba=\bbone$ i transformacji \labelcref{eq:aishibaderivation,eq:aidshibaderivation}.
Wszystkie wyrazy poza wyrazem związanym z potencjałem jednocząstkowym, są niezmiennicze względem transformacji $\shiba$.
Należy zwrócić jedynie uwagę, że poza zmianą znaku w $\mui(\timeNormal)$, transformacja $\shiba$ zmienia $\DeltaSC\to\DeltaSC^*$, co nie wpływa na spektrum energetyczne hamiltonianu.
Ostatecznie pokazano, że
\begin{equation}
    \shiba\hatH_{\trijunction{12}}(\DeltaSC)\shibad=\hatH_{\trijunction{34}}(\DeltaSC).\label{eq:hamiltonianShibaTransformDerivation}
\end{equation}


\ornament

\section*{Równanie~\eqref{eq:shiba6}}
Lokalny operator parzystości transformuje się w następujący sposób:
\begin{flalign}
\shiba\parity_i\shibad &= \shiba(1-2\aid\ai)\shibad = 1-2\shiba\aid\shibad\shiba\ai\shibad =\\
&=1-2(-1)^i\ai(-1)^i\aid = 1-2(1-\aid\ai)=-1+2\aid\ai=-\parity_i,
\end{flalign}
a całkowity operator parzystości
\begin{flalign}
\shiba\parity\shibad=\prod_i \shiba\parity_i\shibad = \prod_i (-1)\parity_i = (-1)^{\sites}\parity = -\parity,&&
\end{flalign}
gdzie skorzystano z relacji antykomutacji operatorów fermionowych $\ai\aid+\aid\ai=1$, unitarności $\shibad\shiba=\bbone$ i transformacji \labelcref{eq:aishibaderivation,eq:aidshibaderivation} oraz założenia nieparzystego $\sites$.

\ornament

\section*{Równanie~\labelcref{eq:shiba7,eq:shiba8}}

Zagadnienie własne hamiltonianu $\hatH_{\trijunction{12}}$:
\begin{align}
    \hatH_{\trijunction{12}}(\DeltaSC)\qstate n &= \Energy_n \qstate n,\\
    \shiba\hatH_{\trijunction{12}}(\DeltaSC)\qstate n &= \Energy_n \shiba\qstate n,\\
    \shiba\hatH_{\trijunction{12}}(\DeltaSC)\shibad\shiba\qstate n &= \Energy_n \shiba\qstate n,\\
    \hatH_{\trijunction{34}}(\DeltaSC^*)\shiba\qstate n &= \Energy_n \shiba\qstate n,
\end{align}
gdzie skorzystano z transformacji \eqref{eq:hamiltonianShibaTransformDerivation}, unitarności $\shibad\shiba=\bbone$.
Ostatnie równanie to zagadnienie własne hamiltonianu $\hatH_{\trijunction{34}}(\DeltaSC^*)$, a stanem własnym jest $\shiba\qstate n$.

Stan własny $\qstate n$ ma określoną parzystość $p_n$
\begin{align}
\parity \qstate n &= p_n \qstate n,
\end{align}
a stan własny $\shiba\qstate n$ ma parzystość $p_n'$
\begin{align}
\parity \shiba\qstate n &= p_n' \shiba\qstate n,\\
\shibad\parity \shiba\qstate n &= p_n' \shibad\shiba\qstate n,\\
-\parity\qstate n &= p_n' \qstate n,\\
\parity\qstate n &= -p_n' \qstate n.
\end{align}
Na podstawie tych rozważań wynika, że stan $\shiba \qstate n$ ma parzystość $p_n'=-p_n$ która  jest przeciwna do parzystości stanu $\qstate n$.
Skorzystano tutaj z postaci transformacji odwrotnej 
\begin{flalign}
\shibad \parity \shiba = -\shibad (\shiba\parity\shibad)\shiba = -\parity.
\end{flalign}
    


\ornament


\section*{Równanie~\eqref{eq:qubitPhaseFactorsGained}}

\begin{flalign}
    \qstate{\psi(2\timeTotal)}  &=  
    \eee^{\iu (\barPhiDyni{e,\,\trijunction{12}}+
     \barPhiDyni{e,\,\trijunction{34}}+
     \barPhiGeoi{e,\,\trijunction{12}} +
    \barPhiGeoi{e,\,\trijunction{34}})} 
    a_0
    \qstate{0}_{\timeNormal=0}+\nonumber\\
    &\qquad\qquad
    \eee^{\iu (\barPhiDyni{o,\,\trijunction{12}}+
    \barPhiDyni{o,\,\trijunction{34}} +
    \barPhiGeoi{o,\,\trijunction{12}} +
    \barPhiGeoi{o,\,\trijunction{34}})} 
    a_1
    \qstate{1}_{\timeNormal=0}=\\
    &=
     \eee^{\iu (\barPhiDyni{e,\,\trijunction{12}}+
     \barPhiDyni{e,\,\trijunction{34}}+
     \barPhiGeoi{e,\,\trijunction{12}} +
    \barPhiGeoi{e,\,\trijunction{34}})} \nonumber\\
    &\qquad\qquad\left(
    a_0 \qstate 0_{\timeNormal=0} + 
        \eee^{-\iu (\barDeltaPhiDyni{o,\,\trijunction{12}}+
    \barDeltaPhiDyni{o,\,\trijunction{34}} +
    \barDeltaPhiGeoi{o,\,\trijunction{12}} +
    \barDeltaPhiGeoi{o,\,\trijunction{34}})} 
    a_1
    \qstate{1}_{\timeNormal=0}
    \right)=\\
    &=\eee^{\iu\chi}\phaseGate\qstate{\psi}
    ,
\end{flalign}



\ornament