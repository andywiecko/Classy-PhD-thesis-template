\chapter{Dwuqubitowy układ z zerowymi modami Majorany}\label{chap:MZMCNOT}

W tym krótkim dodatku zaprezentowano układ zawierający sześć \MZM, który może posłużyć za model układu dwuqubitowego.
W odróżnieniu od analizy przeprowadzonej w~sekcji~\ref{sec:quantumGates}, w takim układzie można skonstruować bramki dwuqubitowe.\\


Układ zawierający sześć \MZM\ musi posiadać $2^{6/2}=8$ zdegenerowanych stanów podstawowych, po cztery stany z określoną parzystością.
Tak jak napisano w sekcji~\ref{sec:quantumGates}, przejścia pomiędzy stanami o różnej parzystości są niedozwolone.
Analizę w tym rozdziale przeprowadzono dla stanów nieparzystych: $\qstate{eoe},\,\qstate{eeo},\,\qstate{oee},\,\qstate{ooo}$.\footnote{Oczywiście identyczną analizę można przeprowadzono dla stanów parzystych: $\qstate{eee},\,\qstate{ooe},\,\qstate{oeo},\,\qstate{eoo}$.}
Bazę  dwuqubitowego układu można ponumerować w następujący sposób:
\begin{align}
\qstate{eoe}& \to \qstate{00},\\
\qstate{eeo}& \to \qstate{01},\\
\qstate{oee}& \to \qstate{10},\\
\qstate{ooo}& \to \qstate{11}.
\end{align}
W odróżnieniu od sekcji~\ref{sec:quantumGates}, układ wraz z operacjami wyplatania tworzy grupę $\braidGroupi 6$.
Ta grupa posiada 10 generatorów $\braidOperatori_1,\dots,\braidOperatori_5,\braidOperatori_1^\dagger,\dots,\braidOperatori_5^\dagger$.
Następnie przeprowadzono analizę działania wybranych operatorów wyplatania oraz ich złożeń na stany bazowe układu dwuqubitowego i odtworzono część wyników jakie otrzymano dla układu reprezentującego pojedynczy qubit.
Na potrzeby dalszej reprezentacji działania poszczególnych operatorów wyplatania i~ich złożeń, przyjęto następującą postać wektorów bazowych:
\begin{equation}
\qstate{00} = 
\left[
\begin{array}{c}
1\\0\\0\\0
\end{array}\right],\,\,
\qstate{01} = 
\left[
\begin{array}{c}
0\\1\\0\\0
\end{array}\right],\,\,
\qstate{10} = 
\left[
\begin{array}{c}
0\\0\\1\\0
\end{array}\right],\,\,
\qstate{11} = 
\left[
\begin{array}{c}
0\\0\\0\\1
\end{array}\right].
\end{equation}
Operatory $\braidOperatori_1$ oraz $\braidOperatori_5$ mają podobne działanie, ale funkcjonują dla różnych qubitów
  w~układzie dwuqubitowym.
Można to w prosty sposób pokazać analogicznie jak pokazano to w sekcji~\ref{sec:quantumGates}.
Poniżej zaprezentowano postać operatorów $\braidOperatori_1$ oraz $\braidOperatori_5$:
\begin{equation}
\braidOperatori_1 = 
\left[
\begin{array}{cccc}
\exp(-\iu\tfrac\pi4)&0&0&0\\
0&\exp(-\iu\tfrac\pi4)&0&0\\
0&0&\exp(+\iu\tfrac\pi4)&0\\
0&0&0&\exp(+\iu\tfrac\pi4)\\
\end{array}\right],
\end{equation}
\begin{equation}
\braidOperatori_5=\left[
\begin{array}{cccc}
\exp(-\iu\tfrac\pi4)&0&0&0\\
0&\exp(+\iu\tfrac\pi4)&0&0\\
0&0&\exp(-\iu\tfrac\pi4)&0\\
0&0&0&\exp(+\iu\tfrac\pi4)\\
\end{array}\right].
\end{equation}
Ich podwójne działanie można zapisać w następujący sposób:
\begin{align}
\zgate_1\equiv \braidOperatori_1^2 & = -\iu\paulii^z_1 \equiv -\iu( \paulii^z \,\kron\, \bbone),\\
\zgate_2\equiv \braidOperatori_5^2 & = -\iu\paulii^z_2 \equiv  -\iu(\bbone\,\kron\,\paulii^z).
\end{align}
Interesujące jest działanie generatora $\braidOperatori_3$
\begin{equation}
\braidOperatori_3=\left[
\begin{array}{cccc}
\exp(+\iu\tfrac\pi4)&0&0&0\\
0&\exp(-\iu\tfrac\pi4)&0&0\\
0&0&\exp(-\iu\tfrac\pi4)&0\\
0&0&0&\exp(+\iu\tfrac\pi4)\\
\end{array}\right].
\end{equation}
Podwójne działanie takiego operatora $\braidOperatori_3$ jest równoważne z działaniem bramki dwuqubitowej Isinga $\zzgate$
\begin{equation}
\zzgate_{12}\equiv \braidOperatori_3^2 = -\iu \braidOperatori_1^2\braidOperatori_5^2 = \iu \paulii_1^z\paulii_2^z \equiv \iu (\paulii^z\,\kron\,\paulii^z).
\end{equation}
Porównanie realizacji operacji $\braidOperatori_1^2,\,\braidOperatori_3^2,\,\braidOperatori_5^2$ można znaleźć na rysunku \ref{fig:z1z1zz12}.
Analogicznie można przeprowadzić analizę dla pozostałych generatorów $\braidOperatori_2$ oraz $\braidOperatori_4$, które można przedstawić w następującej postaci:
\begin{equation}
    \braidOperatori_2=\frac1{\sqrt2}\left[
\begin{array}{cccc}
1&0&-\iu&0\\
0&1&0&-\iu\\
-\iu&0&1&0\\
0&-\iu&0&1\\
\end{array}\right],
\end{equation}
\begin{equation}
    \braidOperatori_4=\frac1{\sqrt2}\left[
\begin{array}{cccc}
1&-\iu&0&0\\
-\iu&1&0&0\\
0&0&1&-\iu\\
0&0&-\iu&1\\
\end{array}\right].
\end{equation}
Podwójne działanie takich operatorów odpowiada realizacjom bramek $\xgate$ na odpowiednich qubitach:
\begin{align}
    \xgate_1 &= \braidOperatori_2^2 = -\iu (\paulii_1^x)\equiv -\iu (\paulii^x\,\kron\,\bbone),\\
    \xgate_2 &= \braidOperatori_4^2 = -\iu (\paulii_2^x)\equiv -\iu (\bbone\,\kron\,\paulii^x).
\end{align}
Realizację operacji $\braidOperatori_2^2$ oraz $\braidOperatori_4^2$ można znaleźć na rysunku~\ref{fig:x1x2}.


\begin{figure}
\begin{tabularx}{\textwidth}{  lcc  lcc lcc}
(a) & & & (b) & & & (c) \\
\begin{minipage}{0.2\textwidth}
  \Qcircuit @C=1em @R=.7em {& \gate{\zgate} & \qw \\& \qw & \qw}\end{minipage} &
  $\equiv$  &  
  \begin{minipage}{0.2\textwidth}\begin{tikzpicture}
\braid[yscale=-0.5,xscale=0.75,number of strands=6,
style strands={1}{red,very thick},
style strands={2}{blue,very thick},
style strands={3}{green,very thick},
style strands={4}{orange,very thick},
style strands={5}{magenta,very thick},
style strands={6}{cyan,very thick},
rotate=90
](braid)
a_1 a_1
;
\node[at=(braid-1-s),left ] {$\Gammaii_1$};
\node[at=(braid-2-s),left ] {$\Gammaii_2$};
\node[at=(braid-3-s),left ] {$\Gammaii_3$};
\node[at=(braid-4-s),left ] {$\Gammaii_4$};
\node[at=(braid-5-s),left ] {$\Gammaii_5$};
\node[at=(braid-6-s),left ] {$\Gammaii_6$};

\end{tikzpicture}
\end{minipage} &
 
 \begin{minipage}{0.2\textwidth}
\Qcircuit @C=1em @R=.7em {& \multigate{1}{\zgate\zgate} & \qw \\& \ghost{\zzgate}& \qw }\end{minipage} &
 $\equiv$ & 
 \begin{minipage}{0.2\textwidth}
 \begin{tikzpicture}
\braid[yscale=-0.5,xscale=0.75,number of strands=6,
style strands={1}{red,very thick},
style strands={2}{blue,very thick},
style strands={3}{green,very thick},
style strands={4}{orange,very thick},
style strands={5}{magenta,very thick},
style strands={6}{cyan,very thick},
rotate=90
](braid)
a_3 a_3
;
\node[at=(braid-1-s),left ] {$\Gammaii_1$};
\node[at=(braid-2-s),left ] {$\Gammaii_2$};
\node[at=(braid-3-s),left ] {$\Gammaii_3$};
\node[at=(braid-4-s),left ] {$\Gammaii_4$};
\node[at=(braid-5-s),left ] {$\Gammaii_5$};
\node[at=(braid-6-s),left ] {$\Gammaii_6$};
\end{tikzpicture}
 \end{minipage} 
&
\begin{minipage}{0.2\textwidth}\Qcircuit @C=1em @R=.7em {& \qw & \qw\\& \gate{\zgate} & \qw } \end{minipage}
& $\equiv$ &
\begin{minipage}{0.2\textwidth}\begin{tikzpicture}
\braid[yscale=-0.5,xscale=0.75,number of strands=6,
style strands={1}{red,very thick},
style strands={2}{blue,very thick},
style strands={3}{green,very thick},
style strands={4}{orange,very thick},
style strands={5}{magenta,very thick},
style strands={6}{cyan,very thick},
rotate=90
](braid)
a_5 a_5
;
\node[at=(braid-1-s),left ] {$\Gammaii_1$};
\node[at=(braid-2-s),left ] {$\Gammaii_2$};
\node[at=(braid-3-s),left ] {$\Gammaii_3$};
\node[at=(braid-4-s),left ] {$\Gammaii_4$};
\node[at=(braid-5-s),left ] {$\Gammaii_5$};
\node[at=(braid-6-s),left ] {$\Gammaii_6$};

\end{tikzpicture}\end{minipage}
\end{tabularx}
\caption[Reprezentacja bramek: $Z_1,\,Z_2,\,ZZ_{12}$.]{
(a) Bramka $\zgate_1$ oraz jej reprezentacja $\braidOperatori_1^2$;
(b) Bramka $\zzgate_{12}$ oraz jej reprezentacja $\braidOperatori_3^2$;
(c) Bramka $\zgate_2$ oraz jej reprezentacja $\braidOperatori_5^2$.
}
\label{fig:z1z1zz12}
\end{figure}







\begin{figure}
\hspace{2cm}
\begin{tabularx}{\textwidth}{  lcc  lcc}
(a) & & & (b)  \\
 \begin{minipage}{0.2\textwidth}
  \Qcircuit @C=1em @R=.7em {& \gate{\xgate} & \qw \\& \qw & \qw}\end{minipage} &
  $\equiv$  &  
  \begin{minipage}{0.2\textwidth}\begin{tikzpicture}
\braid[yscale=-0.5,xscale=0.75,number of strands=6,
style strands={1}{red,very thick},
style strands={2}{blue,very thick},
style strands={3}{green,very thick},
style strands={4}{orange,very thick},
style strands={5}{magenta,very thick},
style strands={6}{cyan,very thick},
rotate=90
](braid)
a_2 a_2
;
\node[at=(braid-1-s),left ] {$\Gammaii_1$};
\node[at=(braid-2-s),left ] {$\Gammaii_2$};
\node[at=(braid-3-s),left ] {$\Gammaii_3$};
\node[at=(braid-4-s),left ] {$\Gammaii_4$};
\node[at=(braid-5-s),left ] {$\Gammaii_5$};
\node[at=(braid-6-s),left ] {$\Gammaii_6$};

\end{tikzpicture}\end{minipage} &
  \begin{minipage}{0.2\textwidth}\Qcircuit @C=1em @R=.7em {& \qw & \qw\\& \gate{\xgate} & \qw } \end{minipage}
& $\equiv$ &
\begin{minipage}{0.2\textwidth}\begin{tikzpicture}
\braid[yscale=-0.5,xscale=0.75,number of strands=6,
style strands={1}{red,very thick},
style strands={2}{blue,very thick},
style strands={3}{green,very thick},
style strands={4}{orange,very thick},
style strands={5}{magenta,very thick},
style strands={6}{cyan,very thick},
rotate=90
](braid)
a_4 a_4
;
\node[at=(braid-1-s),left ] {$\Gammaii_1$};
\node[at=(braid-2-s),left ] {$\Gammaii_2$};
\node[at=(braid-3-s),left ] {$\Gammaii_3$};
\node[at=(braid-4-s),left ] {$\Gammaii_4$};
\node[at=(braid-5-s),left ] {$\Gammaii_5$};
\node[at=(braid-6-s),left ] {$\Gammaii_6$};

\end{tikzpicture}\end{minipage}
\end{tabularx}
\caption[Reprezentacja bramek: $X_1,\,X_2$.]{
(a) Bramka $\xgate_1$ oraz jej reprezentacja $\braidOperatori_2^2$;
(b) Bramka $\xgate_{2}$ oraz jej reprezentacja $\braidOperatori_4^2$.
}
\label{fig:x1x2}
\end{figure}



\begin{figure}
\hspace{2cm}
\begin{tabularx}{\textwidth}{  lcc  lcc}
(a) & & & (b)  \\
 \begin{minipage}{0.2\textwidth}
  \Qcircuit @C=1em @R=.7em {& \gate{\hadamard} & \qw \\& \qw & \qw}\end{minipage} &
  $\equiv$  &  
  \begin{minipage}{0.3\textwidth}\begin{tikzpicture}
\braid[yscale=-0.5,xscale=0.75,number of strands=6,
style strands={1}{red,very thick},
style strands={2}{blue,very thick},
style strands={3}{green,very thick},
style strands={4}{orange,very thick},
style strands={5}{magenta,very thick},
style strands={6}{cyan,very thick},
rotate=90
](braid)
a_1 a_2 a_1
;
\node[at=(braid-1-s),left ] {$\Gammaii_1$};
\node[at=(braid-2-s),left ] {$\Gammaii_2$};
\node[at=(braid-3-s),left ] {$\Gammaii_3$};
\node[at=(braid-4-s),left ] {$\Gammaii_4$};
\node[at=(braid-5-s),left ] {$\Gammaii_5$};
\node[at=(braid-6-s),left ] {$\Gammaii_6$};

\end{tikzpicture}\end{minipage} &
  \begin{minipage}{0.2\textwidth}\Qcircuit @C=1em @R=.7em {& \qw & \qw\\& \gate{\hadamard} & \qw } \end{minipage}
& $\equiv$ &
\begin{minipage}{0.3\textwidth}\begin{tikzpicture}
\braid[yscale=-0.5,xscale=0.75,number of strands=6,
style strands={1}{red,very thick},
style strands={2}{blue,very thick},
style strands={3}{green,very thick},
style strands={4}{orange,very thick},
style strands={5}{magenta,very thick},
style strands={6}{cyan,very thick},
rotate=90
](braid)
a_5 a_4 a_5
;
\node[at=(braid-1-s),left ] {$\Gammaii_1$};
\node[at=(braid-2-s),left ] {$\Gammaii_2$};
\node[at=(braid-3-s),left ] {$\Gammaii_3$};
\node[at=(braid-4-s),left ] {$\Gammaii_4$};
\node[at=(braid-5-s),left ] {$\Gammaii_5$};
\node[at=(braid-6-s),left ] {$\Gammaii_6$};

\end{tikzpicture}\end{minipage}
\end{tabularx}
\caption[Reprezentacja bramek: $H_1,\,H_2$.]{
(a) Bramka $\hadamard_1$ oraz jej reprezentacja $\braidOperatori_1\braidOperatori_2\braidOperatori_1$;
(b) Bramka $\hadamard_{2}$ oraz jej reprezentacja $\braidOperatori_5\braidOperatori_4 \braidOperatori_5$.
}
\label{fig:h1h2}
\end{figure}


\begin{figure}
\hspace{2cm}
\begin{tabularx}{\textwidth}{  lcc  lcc}
(a) & & & (b)  \\
 \begin{minipage}{0.2\textwidth}
  \Qcircuit @C=1em @R=.7em {& \gate{\ygate} & \qw \\& \qw & \qw}\end{minipage} &
  $\equiv$  &  
  \begin{minipage}{0.3\textwidth}\begin{tikzpicture}
\braid[yscale=-0.5,xscale=0.75,number of strands=6,
style strands={1}{red,very thick},
style strands={2}{blue,very thick},
style strands={3}{green,very thick},
style strands={4}{orange,very thick},
style strands={5}{magenta,very thick},
style strands={6}{cyan,very thick},
rotate=90
](braid)
a_1 a_2 a_2 a_1^{-1}
;
\node[at=(braid-1-s),left ] {$\Gammaii_1$};
\node[at=(braid-2-s),left ] {$\Gammaii_2$};
\node[at=(braid-3-s),left ] {$\Gammaii_3$};
\node[at=(braid-4-s),left ] {$\Gammaii_4$};
\node[at=(braid-5-s),left ] {$\Gammaii_5$};
\node[at=(braid-6-s),left ] {$\Gammaii_6$};

\end{tikzpicture}\end{minipage} &
  \begin{minipage}{0.2\textwidth}\Qcircuit @C=1em @R=.7em {& \qw & \qw\\& \gate{\ygate} & \qw } \end{minipage}
& $\equiv$ &
\begin{minipage}{0.3\textwidth}\begin{tikzpicture}
\braid[yscale=-0.5,xscale=0.75,number of strands=6,
style strands={1}{red,very thick},
style strands={2}{blue,very thick},
style strands={3}{green,very thick},
style strands={4}{orange,very thick},
style strands={5}{magenta,very thick},
style strands={6}{cyan,very thick},
rotate=90
](braid)
a_5 a_4 a_4 a_5^{-1}
;
\node[at=(braid-1-s),left ] {$\Gammaii_1$};
\node[at=(braid-2-s),left ] {$\Gammaii_2$};
\node[at=(braid-3-s),left ] {$\Gammaii_3$};
\node[at=(braid-4-s),left ] {$\Gammaii_4$};
\node[at=(braid-5-s),left ] {$\Gammaii_5$};
\node[at=(braid-6-s),left ] {$\Gammaii_6$};

\end{tikzpicture}\end{minipage}
\end{tabularx}
\caption[Reprezentacja bramek: $Y_1,\,Y_2$.]{
(a) Bramka $\ygate_1$ oraz jej reprezentacja $\braidOperatori_1\braidOperatori_2^2\braidOperatori_1^\dagger$;
(b) Bramka $\ygate_{2}$ oraz jej reprezentacja $\braidOperatori_5\braidOperatori_4 ^2\braidOperatori_5^\dagger$.
}
\label{fig:y1y2}
\end{figure}



\begin{figure}
    \centering
    \begin{tikzpicture}
\braid[xscale=1.,yscale=-0.65,number of strands=6,
style strands={1}{red,very thick},
style strands={2}{blue,very thick},
style strands={3}{green,very thick},
style strands={4}{orange,very thick},
style strands={5}{magenta,very thick},
style strands={6}{cyan,very thick},
rotate=90
](braid)
a_3 a_4  a_1-a_3-a_5  a_4^{-1} a_3^{-1}
;
\node[at=(braid-1-s),left ] {$\Gammaii_1$};
\node[at=(braid-2-s),left ] {$\Gammaii_2$};
\node[at=(braid-3-s),left ] {$\Gammaii_3$};
\node[at=(braid-4-s),left ] {$\Gammaii_4$};
\node[at=(braid-5-s),left ] {$\Gammaii_5$};
\node[at=(braid-6-s),left ] {$\Gammaii_6$};

\node[right]at (-5,-2.5){\LARGE
\Qcircuit @C=1em @R=.7em {
& \ctrl{1} &  \qw \\
& \targ &   \qw}};
\node[right]at(-2,-2.5){$\equiv$};

\end{tikzpicture}{}
    \caption[Relizacja bramki \texttt{CNOT}.]{Realizacja bramki \CNOT\ oraz jej reprezentacja $\braidOperatori_1\braidOperatori_3\braidOperatori_4\braidOperatori_3\braidOperatori_5\braidOperatori_4^\dagger\braidOperatori_3^\dagger$.}
    \label{fig:cnot}
\end{figure}

\newpage
Korzystając z wymienionych generatorów $\braidOperatori_1$, \dots, $\braidOperatori_5$ oraz ich sprzężeń, można utworzyć inne bramki:
\begin{itemize}

\item bramki Hadamarda $\hadamard$ (przedstawione na rysunku~\ref{fig:h1h2})
  \begin{align}
    \hadamard_1 &= \braidOperatori_1\braidOperatori_2\braidOperatori_1 \equiv - \tfrac{\iu}{\sqrt2}
    \begin{bmatrix}
    1 & 1\\
    1 & -1
    \end{bmatrix}
    \,\kron\,\bbone,\\
        \hadamard_1 &= \braidOperatori_1\braidOperatori_2\braidOperatori_1 \equiv - \tfrac{\iu}{\sqrt2}\, \bbone \,\kron\,
    \begin{bmatrix}
    1 & 1\\
    1 & -1
    \end{bmatrix}
  ,
\end{align}
  
  
  
      \item bramki $\ygate$ (przedstawione na rysunku~\ref{fig:y1y2})
\begin{align}
    \ygate_1 &= \braidOperatori_1\braidOperatori_2^2\braidOperatori_1^\dagger = -\iu (\paulii_1^y)\equiv -\iu (\paulii^y\,\kron\,\bbone),\\
    \ygate_2 &= \braidOperatori_5\braidOperatori_4^2\braidOperatori_5^\dagger = -\iu (\paulii_2^y)\equiv -\iu (\bbone\,\kron\,\paulii^y),
\end{align}
    
    \item bramki $\CNOT$ (przedstawiona na rysunku~\ref{fig:cnot})
    \begin{equation}
        \CNOT = \braidOperatori_1\braidOperatori_3\braidOperatori_4\braidOperatori_3\braidOperatori_5\braidOperatori_4^\dagger\braidOperatori_3^\dagger= \eee^{-\iu\pi/4}\left[
        \begin{array}{cccc}
            1 & 0 & 0 & 0  \\
            0 & 1 & 0 & 0  \\
            0 & 0 & 0 & 1  \\
            0 & 0 & 1 & 0 
        \end{array}\right].
    \end{equation}
    
\end{itemize}

\ornament
