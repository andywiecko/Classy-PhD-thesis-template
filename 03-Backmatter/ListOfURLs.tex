
\chapter{Spis URL}

{
\small
\noindent
\begin{tabular}{l>{\scriptsize}l}
twierdzenie o dwumianie & \url{https://pl.wikipedia.org/wiki/Dwumian_Newtona} \\
big endian & \url{https://en.wikipedia.org/wiki/Endianness} \\
%problem znaku & \url{https://en.wikipedia.org/wiki/Numerical_sign_problem} \\
kombinacyjny system liczbowy & \url{https://en.wikipedia.org/wiki/Combinatorial_number_system} \\
\texttt{\small SOLIDstate} & \url{https://github.com/andywiecko/SOLIDstate} \\
\texttt{\small Armadillo} & \url{http://arma.sourceforge.net/} \\
wrapper & \url{https://en.wikipedia.org/wiki/Wrapper_library} \\
twierdzenie min-max & \url{https://en.wikipedia.org/wiki/Min-max_theorem} \\
metoda punktu środkowego & \url{https://en.wikipedia.org/wiki/Midpoint_method} \\
funkcja Bessela & \url{https://pl.wikipedia.org/wiki/Funkcje_Bessela}\\

termodynamiczna odwrotność temperatury & \url{https://en.wikipedia.org/wiki/Thermodynamic_beta} \\
bramka \texttt{\small CNOT} & \url{https://en.wikipedia.org/wiki/Controlled_NOT_gate} \\
bramka fazowa & \url{https://en.wikipedia.org/wiki/Quantum_logic_gate\#Phase_shift}\\
bramka Hadamarda & \url{https://en.wikipedia.org/wiki/Quantum_logic_gate\#Hadamard_(H)_gate} \\
bramka $X$ & \url{https://en.wikipedia.org/wiki/Quantum_logic_gate\#PauliX_gate} \\
bramka $Y$ & \url{https://en.wikipedia.org/wiki/Quantum_logic_gate\#Pauli-Y_gate} \\
bramka $Z$ & \url{https://en.wikipedia.org/wiki/Quantum_logic_gate\#Pauli-Z_gate} \\
bramka $ZZ$ Isinga & \url{https://en.wikipedia.org/wiki/Quantum_logic_gate\#Ising_(ZZ)_coupling_gate} \\

delta Diraca & \url{https://pl.wikipedia.org/wiki/Delta_Diraca} \\
delta Kroneckera & \url{https://pl.wikipedia.org/wiki/Symbol_Kroneckera}\\
funkcja Heaviside'a & \url{https://pl.wikipedia.org/wiki/Funkcja_skokowa_Heaviside\%E2\%80\%99a}\\
grupa warkoczowa & \url{https://en.wikipedia.org/wiki/Braid_group} \\
iloczyn Kroneckera & \url{https://pl.wikipedia.org/wiki/Iloczyn_Kroneckera} \\
jednostka urojona & \url{https://pl.wikipedia.org/wiki/Jednostka_urojona}\\


liczba Eulera & \url{https://pl.wikipedia.org/wiki/Podstawa_logarytmu_naturalnego}\\
liczba pi & \url{https://pl.wikipedia.org/wiki/Pi}\\
macierze Pauliego & \url{https://pl.wikipedia.org/wiki/Macierze_Pauliego} \\
notacja duże O & \url{https://pl.wikipedia.org/wiki/Asymptotyczne_tempo_wzrostu}\\
stała Plancka & \url{https://en.wikipedia.org/wiki/Planck_constant} \\
suma prosta & \url{https://pl.wikipedia.org/wiki/Suma_prosta_przestrzeni_liniowych}\\
wielomiany Czebyszewa & \url{https://pl.wikipedia.org/wiki/Wielomiany_Czebyszewa} \\



\end{tabular}
}
