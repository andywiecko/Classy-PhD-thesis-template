\newglossaryentry{symb:anihilation}
{
    name=\ensuremath{a_i},
    description={operator anihilacji bezspinowego fermionu},
    type=symbolslist,
    sort=operator anihilacji
}

\newglossaryentry{symb:anticommutator}
{
    name=\ensuremath{\{A,B\}},
    description={$=AB+BA$, antykomutator operatorów},
    type=symbolslist,
    sort = antykomutator
}

\newglossaryentry{symb:beta}
{
    name=\ensuremath{\beta},
    description={\href{https://en.wikipedia.org/wiki/Thermodynamic_beta}{termodynamiczna odwrotność temperatury}},
    type=symbolslist,
    sort=beta
} 
\newglossaryentry{symb:braidOperator}
{
    name=\ensuremath{\mathcal B_i},
    description={operator wyplatania (eng. \textit{braiding operator})},
    type=symbolslist,
    sort=operator wyplatania
} 

\newglossaryentry{symb:rotation}
{
    name=\ensuremath{\mathcal O(\omega)},
    description={macierz obrotu},
    type=symbolslist,
    sort=macierz obrotu
} 

\newglossaryentry{symb:braidGroup}
{
    name=\ensuremath{\mathfrak B_M},
    description={\href{https://en.wikipedia.org/wiki/Braid_group}{$M$-włóknowa grupa warkoczowa Artina (ang. braid group)}},
    type=symbolslist,
    sort = grupa warkoczowa
} 

\newglossaryentry{symb:chemicalPotential}
{
    name=\ensuremath{\mu_i},
    description={potencjał chemiczny},
    type=symbolslist,
    sort = potencjał chemiczny
}

\newglossaryentry{symb:commutator}
{
    name=\ensuremath{[A,B]},
    description={$=AB-BA$, komutator operatorów},
    type=symbolslist,
    sort = komutator
}


\newglossaryentry{symb:creation}
{
    name=\ensuremath{a_i^\dagger},
    description={operator kreacji bezspinowego fermionu},
    type=symbolslist,
    sort = operator kreacji
}

\newglossaryentry{symb:deltaSC}
{
    name=\ensuremath{\Delta_{ij}},
    description={przerwa nadprzewodząca},
    type=symbolslist,
    sort = przerwa nadprzewodząca
}

%\newglossaryentry{symb:diracMatrix}
%{
%    name=\ensuremath{\upgamma^\mu},
%    description={\href{https://en.wikipedia.org/wiki/Gamma_matrices}{macierze Diraca}},
%    type=symbolslist,
%    sort=macierze Diraca
%}

\newglossaryentry{symb:energy}
{
    name=\ensuremath{E},
    description={energia},
    type=symbolslist,
    sort = energia
}

\newglossaryentry{symb:deltaE}
{
    name=\ensuremath{\delta E},
    description={degeneracja (różnica energii) stanów podstawowych z podprzestrzeni z parzystą i nieparzystą liczbą cząstek},
    type=symbolslist,
    sort = energia roznica degeneracja
}

\newglossaryentry{symb:DeltaE}
{
    name=\ensuremath{\Delta E},
    description={szczelina energetyczna},
    type=symbolslist,
    sort = energia roznica szczelina
}

\newglossaryentry{symb:field12}
{
    name=\ensuremath{\Psi},
    description={kwantowe pole cząstek o spinie $\tfrac12$},
    type=symbolslist,
    sort = pole kwantowe
}

\newglossaryentry{symb:hamiltonian}
{
    name=\ensuremath{\hat H},
    description={hamiltonian},
    type=symbolslist,
    sort = hamiltonian
}

\newglossaryentry{symb:hc}
{
    name=\ensuremath{\mathrm{H.c.}},
    description={sprzężenie hermitowskie (eng. \textit{Hermitian conjugate})},
    type=symbolslist,
    sort=sprzężenie hermitowskie
}

\newglossaryentry{symb:hilbertSchmitd}
{
    name=\ensuremath{(A|B)},
    description={iloczyn Hilberta--Schmitda operatorów},
    type=symbolslist,
    sort = iloczyn hilberta-schmitda
}

\newglossaryentry{symb:hilbertSpace}
{
    name=\ensuremath{\mathcal H},
    description={przestrzeń Hilberta},
    type=symbolslist,
    sort = przestrzeń Hilberta
}

\newglossaryentry{symb:hoppingIntegral}
{
    name=\ensuremath{t^0_{ij}},
    description={całka przeskoku},
    type=symbolslist,
    sort=całka przeskoku
}

\newglossaryentry{symb:imagUnit}
{
    name=\ensuremath{\mathrm{i}},
    description={\href{https://pl.wikipedia.org/wiki/Jednostka_urojona}{jednostka urojona}},
    type=symbolslist,
    sort=jednostka urojona
}

\newglossaryentry{symb:kroneckerSymbol}
{
    name=\ensuremath{\delta_{ij}},
    description={\href{https://pl.wikipedia.org/wiki/Symbol_Kroneckera}{delta Kroneckera}},
    type=symbolslist,
    sort=delta Kronecera
}

\newglossaryentry{symb:majoranaOperator}
{
    name=\ensuremath{\gamma_i},
    description={operator Majorany},
    type=symbolslist,
    sort=operator Majorany
}

\newglossaryentry{symb:majoranaBase}
{
    name=\ensuremath{\altmathcal B_{\gamma_i}},
    description={baza operatorów Majorany \gammai},
    type=symbolslist,
    sort=baza operatorów Majorany
}

\newglossaryentry{symb:majoranaMode}
{
    name=\ensuremath{\Gamma_i},
    description={zerowa moda Majorany},
    type=symbolslist,
    sort = zerowa moda Majorany
}

\newglossaryentry{symb:majoranaModeCoef}
{
    name=\ensuremath{\alpha_i},
    description={współczynniki z bazy $\gammai$ zerowej mody Majorany},
    type=symbolslist,
    sort=współczynnik zerowej mody Majorany
}

\newglossaryentry{symb:mass}
{
    name=\ensuremath{m},
    description={masa},
    type=symbolslist,
    sort=masa
}

\newglossaryentry{symb:one}
{
    name=\ensuremath{\mathbb 1},
    description={macierz jednostkowa/operator jednostkowy},
    type=symbolslist,
    sort = macierz jednostkowa
}

\newglossaryentry{symb:zeroVector}
{
    name=\ensuremath{\mathbb 0},
    description={macierz/operator/wektor zerowy},
    type=symbolslist,
    sort = macierz zerowa
}

\newglossaryentry{symb:realNumbers}
{
    name=\ensuremath{\mathbb R},
    description={zbiór liczb rzeczywistych},
    type=symbolslist,
    sort=zbiór liczb rzeczywistych
}

\newglossaryentry{symb:complexNumbers}
{
    name=\ensuremath{\mathbb C},
    description={zbiór liczb zespolonych},
    type=symbolslist,
    sort=zbiór liczb zespolonych
}

\newglossaryentry{symb:sites}
{
    name=\ensuremath{L},
    description={liczba węzłów},
    type=symbolslist,
    sort=liczba węzłów
}

\newglossaryentry{symb:sitesChain}
{
    name=\ensuremath{\ell},
    description={liczba węzłów pojedynczego łańcucha trójzłącza},
    type=symbolslist,
    sort=liczba węzłów łańcucha
}

\newglossaryentry{symb:particles}
{
    name=\ensuremath{N},
    description={liczba cząstek},
    type=symbolslist,
    sort=liczba cząstek
}

\newglossaryentry{symb:parity}
{
    name=\ensuremath{\mathcal P},
    description={operator parzystości},
    type=symbolslist,
    sort=operator parzystości
}

\newglossaryentry{symb:particleNumber}
{
    name=\ensuremath{\hat n_i},
    description={operator liczby cząstek},
    type=symbolslist,
    sort=operator liczby cząstek
}

\newglossaryentry{symb:particleNumberModified}
{
    name=\ensuremath{\textcircumacute{\(n\)}_{i}},
    description={operator liczby cząstek $\ni$ pomniejszony o $\tfrac12$},
    type=symbolslist,
    sort=operator liczby cząstek pomniejszony
}

\newglossaryentry{symb:totalParticleNumber}
{
    name=\ensuremath{\hat N},
    description={operator liczby wszystkich cząstek},
    type=symbolslist,
    sort=operator liczby wszystkich cząstek
}


\newglossaryentry{symb:pauli}
{
    name=\ensuremath{\sigma^i},
    description={\href{https://pl.wikipedia.org/wiki/Macierze_Pauliego}{macierze Pauliego} (w przestrzeni spinów)},
    type=symbolslist,
    sort=macierze Pauliego
}

\newglossaryentry{symb:pauliParticle}
{
    name=\ensuremath{\tau^i},
    description={macierze Pauliego (w przestrzeni cząstka--dziura)},
    type=symbolslist,
    sort=macierze Pauliego 2
}

\newglossaryentry{symb:zeeman}
{
    name=\ensuremath{V_Z},
    description={oddziaływanie Zeemana},
    type=symbolslist,
    sort=zeeman
}


\newglossaryentry{symb:trace}
{
    name=\ensuremath{\mathrm{Tr}},
    description={ślad (eng. \textit{trace})},
    type=symbolslist,
    sort=slad
}

\newglossaryentry{symb:propagator}
{
    name=\ensuremath{\mathcal{U}},
    description={propagator},
    type=symbolslist,
    sort=propagator
}

\newglossaryentry{symb:state}
{
    name=\ensuremath{|\psi\rangle},
    description={stan kwantowy},
    type=symbolslist,
    sort=stan kwantowy
}


\newglossaryentry{symb:tensor}
{
    name=\ensuremath{\otimes},
    description={\href{https://pl.wikipedia.org/wiki/Iloczyn_Kroneckera}{iloczyn Kroneckera}},
    type=symbolslist,
    sort= iloczyn Kroneckera
}

\newglossaryentry{symb:directsum}
{
    name=\ensuremath{\oplus},
    description={\href{https://pl.wikipedia.org/wiki/Suma_prosta_przestrzeni_liniowych}{suma prosta}},
    type=symbolslist,
    sort= suma prosta
}

\newglossaryentry{symb:zgate}
{
    name=\ensuremath{Z},
    description={\href{https://en.wikipedia.org/wiki/Quantum_logic_gate\#Pauli-Z_(\%7F'\%22\%60UNIQ--postMath-00000022-QINU\%60\%22'\%7F)_gate}{bramka $Z$}},
    type=symbolslist,
    sort=bramka Z
}

\newglossaryentry{symb:ygate}
{
    name=\ensuremath{Y},
    description={\href{https://en.wikipedia.org/wiki/Quantum_logic_gate\#Pauli-Y_gate}{bramka $Y$}},
    type=symbolslist,
    sort= bramka Y
}

\newglossaryentry{symb:xgate}
{
    name=\ensuremath{X},
    description={\href{https://en.wikipedia.org/wiki/Quantum_logic_gate\#Pauli-X_gate}{bramka $X$}},
    type=symbolslist,
    sort = bramka X
}

\newglossaryentry{symb:hadamard}
{
    name=\ensuremath{\textbf{\textit{\textsf H}}},
    description={\href{https://en.wikipedia.org/wiki/Quantum_logic_gate\#Hadamard_(H)_gate}{bramka Hadamarda}},
    type=symbolslist,
    sort=bramka Hadamarda
}

\newglossaryentry{symb:CNOT}
{
    name=\ensuremath{\mathtt{CNOT}},
    description={\href{https://en.wikipedia.org/wiki/Controlled_NOT_gate}{bramka \texttt{CNOT}}},
    type=symbolslist,
    sort=bramka CNOT
}

\newglossaryentry{symb:zzgate}
{
    name=\ensuremath{ZZ},
    description={\href{https://en.wikipedia.org/wiki/Quantum_logic_gate\#Ising_(ZZ)_coupling_gate
}{bramka $ZZ$ Isinga}},
    type=symbolslist,
    sort=bramka ZZ
    }


\newglossaryentry{symb:phaseGate}
{
    name=\ensuremath{R(\theta)},
    description={\href{https://en.wikipedia.org/wiki/Quantum_logic_gate\#Phase_shift_(\%7F'\%22\%60UNIQ--postMath-00000032-QINU\%60\%22'\%7F)_gates}{bramka fazowa}},
    type=symbolslist,
    sort=bramka fazowa
}

\let\piOld\pi
\newglossaryentry{symb:pi}
{
    name=\ensuremath{\piOld},
    description={\href{https://pl.wikipedia.org/wiki/Pi}{liczba pi}},
    type=symbolslist,
    sort=liczba pi
}

 
\newglossaryentry{symb:hbar}
{
    name=\ensuremath{\hbar},
    description={\href{https://en.wikipedia.org/wiki/Planck_constant}{stała Plancka} podzielona prze $2\pi$},
    type=symbolslist,
    sort=stala plancka
}


\newglossaryentry{symb:e}
{
    name=\ensuremath{\mathrm e},
    description={\href{https://pl.wikipedia.org/wiki/Podstawa_logarytmu_naturalnego}{liczba Eulera}},
    type=symbolslist,
    sort=liczba e
}

\newglossaryentry{symb:time}
{
    name=\ensuremath{t},
    description={czas},
    type=symbolslist,
    sort=czas
}

\newglossaryentry{symb:timeTotal}
{
    name=\ensuremath{\mathcal T},
    description={całkowity czas ewolucji},
    type=symbolslist,
    sort=czas całkowity
}

\newglossaryentry{symb:timePartial}
{
    name=\ensuremath{T},
    description={czas segmentu ewolucji},
    type=symbolslist,
    sort=czas częściowy
}

\newglossaryentry{symb:trijunction}
{
    name=\ensuremath{J_{12}},
    description={trójzłącze},
    type=symbolslist,
    sort=trójzłącze
}


\newglossaryentry{symb:timeTau}
{
    name=\ensuremath{\tau},
    description={skala czasowa},
    type=symbolslist,
    sort=czas skala
}

\newglossaryentry{symb:timeTauI}
{
    name=\ensuremath{\tau_{I}},
    description={skala czasowa związana z mechanizmem relaksacji na skutek oddziaływań wielociałowych},
    type=symbolslist,
    sort=czas skala relaksacja oddzialywania wielocialwe
}

\newglossaryentry{symb:timeTauM}
{
    name=\ensuremath{\tau_{M}},
    description={skala czasowa związana z mechanizmem relaksacji na skutek oddziaływań \MZM--\MZM},
    type=symbolslist,
    sort=czas skala relaksacja oddzialywania mzm mzm
}

\newglossaryentry{symb:Vij}
{
    name=\ensuremath{V_{ij}},
    description={potencjał oddziaływania fermion-fermion},
    type=symbolslist,
    sort=potencjal oddzialywania
}

\newglossaryentry{symb:bigO}
{
    name=\ensuremath{\altmathcal{O}},
    description={\href{https://pl.wikipedia.org/wiki/Asymptotyczne_tempo_wzrostu}{notacja duże O}},
    type=symbolslist,
    sort = O
}

\newglossaryentry{symb:timeAverage}
{
    name=\ensuremath{\bar A},
    description={średnia po nieskończonym czasie operatora $A$},
    type=symbolslist,
    sort = srednia po czasie
}


\newglossaryentry{symb:timeTauAverage}
{
    name=\ensuremath{\bar A^{\tau}},
    description={średnia po skali czasowej $\timeTau$ operatora $A$ [definicja \eqref{eq:timeTauAvgDef}]},
    type=symbolslist,
    sort = srednia po skali czasowej
}

\newglossaryentry{symb:conductance}
{
    name=\ensuremath{G},
    description={przewodność różniczkowa},
    type=symbolslist,
    sort = przewodnosc rozniczkowa
}


\newglossaryentry{symb:heaviside}
{
    name=\ensuremath{\Theta(x)},
    description={\href{https://pl.wikipedia.org/wiki/Funkcja_skokowa_Heaviside\%E2\%80\%99a}{funkcja Heaviside'a}},
    type=symbolslist,
    sort = funkcja heaviside
}

\newglossaryentry{symb:corrLength}
{
    name=\ensuremath{\xi},
    description={długość korelacji},
    type=symbolslist,
    sort = dlugosc korelacji
}

\newglossaryentry{symb:lambda}
{
    name=\ensuremath{\lambda},
    description={korelacja $(\barGammaii|\Gammaii)$},
    type=symbolslist,
    sort = korelacja gamma gamma
}

\newglossaryentry{symb:chebyshev}
{
    name=\ensuremath{T_k(x)},
    description={\href{https://pl.wikipedia.org/wiki/Wielomiany_Czebyszewa}{wielomian Czebyshewa} stopnia $k$},
    type=symbolslist,
    sort = wielomian czebyszewa
}

\newglossaryentry{symb:bessel}
{
    name=\ensuremath{J_k(x)},
    description={\href{https://pl.wikipedia.org/wiki/Funkcje_Bessela\#Funkcje_Bessela_pierwszego_rodzaju}{funkcja Bessla pierwszego rodzaju rzędu $k$}},
    type=symbolslist,
    sort = funkcja bessla
}


\newglossaryentry{symb:corrMatrix}
{
    name=\ensuremath{K},
    description={macierz korelacji $(\bargammai|\gammaj)$},
    type=symbolslist,
    sort = macierz korelacji gamma gamma
}

\newglossaryentry{symb:diracDelta}
{
    name=\ensuremath{\delta(x)},
    description={\href{https://pl.wikipedia.org/wiki/Delta_Diraca}{delta Diraca}},
    type=symbolslist,
    sort = delta diraca
}

\newglossaryentry{symb:chrono}
{
    name=\ensuremath{\hat{\altmathcal{T}}},
    description={operator chronologiczny},
    type=symbolslist,
    sort= operator chronologiczny
}


\newglossaryentry{symb:berry}
{
    name=\ensuremath{\phi_{\mathrm{Berry}}},
    description={faza Berry'ego},
    type=symbolslist,
    sort= faza berryego
}

\newglossaryentry{symb:AA}
{
    name=\ensuremath{\phi_{\mathrm{AA}}},
    description={faza Aharonova--Anandana},
    type=symbolslist,
    sort= faza Aharonova--Anandana
}

\newglossaryentry{symb:geo}
{
    name=\ensuremath{\phi_{\mathrm{geo}}},
    description={faza geometryczna},
    type=symbolslist,
    sort= faza geometryczna
}

\newglossaryentry{symb:dyn}
{
    name=\ensuremath{\phi_{\mathrm{dyn}}},
    description={faza dynamiczna},
    type=symbolslist,
    sort= faza dynamiczna
}

\newglossaryentry{symb:ex}
{
    name=\ensuremath{\phi_{\mathrm{ex}}},
    description={faza wymiany},
    type=symbolslist,
    sort= faza wymiany
}

\newglossaryentry{symb:DeltaPhi}
{
    name=\ensuremath{\Delta\phi},
    description={różnica odpowiednich faz z podprzestrzeni z parzystą i nieparzystą liczbą cząstek},
    type=symbolslist,
    sort= faza różnica 
}

\newglossaryentry{symb:epsilon}
{
    name=\ensuremath{\epsilon},
    description={błąd wyplatania},
    type=symbolslist,
    sort= błąd wyplatania
}

\newglossaryentry{symb:shiba}
{
    name=\ensuremath{U_{\mathrm{sh}}},
    description={transformacja Shiba'y},
    type=symbolslist,
    sort= transformacja Shiba
}

\newglossaryentry{symb:PhiSC}
{
    name=\ensuremath{\varphi^{\mathrm{sc}}_{ij}},
    description={faza parametru nadprzewodnictwa},
    type=symbolslist,
    sort= faza parametru nadprzewodnictwa
}

\newglossaryentry{symb:fidelity}
{
    name=\ensuremath{w_{\mathrm{loss}}},
    description={straty wiarygodności (eng. \textit{fidelity loss})},
    type=symbolslist,
    sort= fidelity
}

\newglossaryentry{symb:MZMoverlap}
{
    name=\ensuremath{\Omega},
    description={całkowite przekrycie \MZM},
    type=symbolslist,
    sort= przekrycie mzm
}
